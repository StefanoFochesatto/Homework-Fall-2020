%%%%%%%%%%%%%%%%%%%%%%%%%%%%%%%%%%%%%%%%%%%%%%%%%%%%%%%%%%%%%%%%%%%%%%%%%%%%%%%%%%%%%%%
%%%%%%%%%%%%%%%%%%%%%%%%%%%%%%%%%%%%%%%%%%%%%%%%%%%%%%%%%%%%%%%%%%%%%%%%%%%%%%%%%%%%%%%
% 
% This top part of the document is called the 'preamble'.  Modify it with caution!
%
% The real document starts below where it says 'The main document starts here'.

\documentclass[12pt]{article}

\usepackage{amssymb,amsmath,amsthm}
\usepackage[top=1in, bottom=1in, left=1.25in, right=1.25in]{geometry}
\usepackage{fancyhdr}
\usepackage{enumerate}

% Comment the following line to use TeX's default font of Computer Modern.
\usepackage{times,txfonts}

\newtheoremstyle{homework}% name of the style to be used
  {18pt}% measure of space to leave above the theorem. E.g.: 3pt
  {12pt}% measure of space to leave below the theorem. E.g.: 3pt
  {}% name of font to use in the body of the theorem
  {}% measure of space to indent
  {\bfseries}% name of head font
  {:}% punctuation between head and body
  {2ex}% space after theorem head; " " = normal interword space
  {}% Manually specify head
\theoremstyle{homework} 

% Set up an Exercise environment and a Solution label.
\newtheorem*{exercisecore}{Exercise \@currentlabel}
\newenvironment{exercise}[1]
{\def\@currentlabel{#1}\exercisecore}
{\endexercisecore}

\newcommand{\localhead}[1]{\par\smallskip\noindent\textbf{#1}\nobreak\\}%
\newcommand\solution{\localhead{Solution:}}

%%%%%%%%%%%%%%%%%%%%%%%%%%%%%%%%%%%%%%%%%%%%%%%%%%%%%%%%%%%%%%%%%%%%%%%%
%
% Stuff for getting the name/document date/title across the header
\makeatletter
\RequirePackage{fancyhdr}
\pagestyle{fancy}
\fancyfoot[C]{\ifnum \value{page} > 1\relax\thepage\fi}
\fancyhead[L]{\ifx\@doclabel\@empty\else\@doclabel\fi}
\fancyhead[C]{\ifx\@docdate\@empty\else\@docdate\fi}
\fancyhead[R]{\ifx\@docauthor\@empty\else\@docauthor\fi}
\headheight 15pt

\def\doclabel#1{\gdef\@doclabel{#1}}
\doclabel{Use {\tt\textbackslash doclabel\{MY LABEL\}}.}
\def\docdate#1{\gdef\@docdate{#1}}
\docdate{Use {\tt\textbackslash docdate\{MY DATE\}}.}
\def\docauthor#1{\gdef\@docauthor{#1}}
\docauthor{Use {\tt\textbackslash docauthor\{MY NAME\}}.}
\makeatother

% Shortcuts for blackboard bold number sets (reals, integers, etc.)
\newcommand{\Reals}{\ensuremath{\mathbb R}}
\newcommand{\Nats}{\ensuremath{\mathbb N}}
\newcommand{\Ints}{\ensuremath{\mathbb Z}}
\newcommand{\Rats}{\ensuremath{\mathbb Q}}
\newcommand{\Cplx}{\ensuremath{\mathbb C}}
%% Some equivalents that some people may prefer.
\let\RR\Reals
\let\NN\Nats
\let\II\Ints
\let\CC\Cplx

%%%%%%%%%%%%%%%%%%%%%%%%%%%%%%%%%%%%%%%%%%%%%%%%%%%%%%%%%%%%%%%%%%%%%%%%%%%%%%%%%%%%%%%
%%%%%%%%%%%%%%%%%%%%%%%%%%%%%%%%%%%%%%%%%%%%%%%%%%%%%%%%%%%%%%%%%%%%%%%%%%%%%%%%%%%%%%%
% 
% The main document start here.

% The following commands set up the material that appears in the header.
\doclabel{Math 401: Homework 5}
\docauthor{Stefano Fochesatto}
\docdate{September 29, 2020}

\begin{document}

\begin{exercise}{Supplemental 1} 
Suppose $(a_n)\to a$ and $a\neq 0$.
Show that there exists $N\in \Nats$
such that if $n\ge N$, then $a_n\neq 0$.
\end{exercise}
\begin{proof}
  Suppose that the sequence $(a_n) \to a$ and $a \neq 0$. Since the sequence $(a_n)$ converges we know that for all, 
  $\epsilon \in \Reals$, where $\epsilon < 0$ there exists an $N\in \Nats$ such that for all $n \geq N$, 
  \begin{equation*}
    |a_n - a| < \epsilon.
  \end{equation*} 
Consider an $\epsilon < a$ then there exists some $N\in \Nats$ such that for all $n \geq N$,
\begin{align*}
  &|a_n - a| < \epsilon,\\
  a - \epsilon < &a_n < a + \epsilon,\\
  0 < &a_n < a + \epsilon.
\end{align*} 
Thus we have shown that there exists $N\in \Nats$
such that if $n\ge N$, then $a_n\neq 0$.
\end{proof}
\vspace{.5in}









\begin{exercise}{Supplemental 2}

\begin{enumerate}
	\item Show that if $a,b\ge 0$ and $a>b$, then $\sqrt{a}>\sqrt{b}$.\\
	
  \begin{proof}Let that $a,b\ge 0$, now suppose $\sqrt{a} \le \sqrt{b}$. Through some algebra,
    \begin{align*}
      a &= \sqrt{a}\sqrt{a}\\
      &\le \sqrt{a}\sqrt{b}\\
      &\le \sqrt{b}\sqrt{b}\\
      &= b
    \end{align*}
    Thus we have shown that $a \le b$, and thus by contrapositive if  $a,b\ge 0$ and $a>b$, then $\sqrt{a}>\sqrt{b}$.
  \end{proof}
  \vspace{.25in}




\item Exercise 2.3.1(a) If $(x_n) \to 0$, show that $\sqrt{(x_n)} \to 0$\\

\begin{proof}
  Suppose the convergent sequence $(x_n)$ such that $(x_n) \to 0$. Recall by the definition of convergent
  for all $\epsilon > 0$ we know that there exists an $N \in \Nats$ such that when $n\geq N$, 
  \begin{equation*}
    |x_n| < \epsilon.
  \end{equation*}
  Note that since this inequality is true for all $\epsilon > 0$, its also true for $\epsilon^2$ which leaves us with, 
  \begin{align*}
    x_n &< \epsilon^2\\
    \sqrt{x_n} &< \epsilon
  \end{align*}
  Thus we have shown that $\sqrt{(x_n)} \to 0$.
\end{proof}

\end{enumerate}
\end{exercise}
\vspace{.5in}










\begin{exercise}{2.3.3} Show that if $x_n \le y_n \le z_n$ for all $n \in \Nats$, 
  and if $\lim x_n = \lim z_n = l$ then $\lim y_n = l$ as well. \\

  \begin{proof}
    Suppose that $x_n \le y_n \le z_n$ for all $n \in \Nats$, and that $\lim x_n = \lim z_n = l$. Let $\epsilon > 0$,
    since both $x_n$ and $z_n$ converge we know that there exists $N_x, N_z \in \Nats$ such that for all $n_x \geq N_x$, $n_z \geq N_z$, the following are true,
    \begin{equation*}
      |x_{n_x} - l| \le \epsilon 
    \end{equation*}   
    \begin{equation*}
      |z_{n_z} - l| \le \epsilon 
    \end{equation*}
    Now let $N = max\{N_x, N_z\}$, to ensure that the above inequalities apply. Therefore for all $n \geq N$,
    \begin{equation*}
      -\epsilon < x_n - l <z_n - l <\epsilon.
    \end{equation*}   
    Recall, that through algebra we get,
    \begin{align*}
      x_n \le &y_n \le z_n,\\
      x_n - l \le &y_n - l \le z_n - l.
    \end{align*}
    Therefore the following is true,
    \begin{align*}
       -\epsilon < x_n - l \le &y_n - l \le z_n - l <\epsilon,\\
        -\epsilon < &y_n - l < \epsilon,\\
        |y_n - l| &< \epsilon.
    \end{align*}
 Thus we have shown that $\lim y_n = l$.
  \end{proof}
\end{exercise}
\vspace{.5in}







\begin{exercise}{2.3.10} Consider the following list of conjectures. Provide a short proof for those that sre true and a 
  counterexample for any that are false. 
  \begin{enumerate}
    \item If $\lim (a_n - b_n) = 0$, then $\lim a_n = \lim b_n$\\
    
    \begin{proof}
      Consider $a_n = (-1)^{n+1}$ and $b_n = (-1)^n$. Clearly the following equation is true over all values of $n$,
      \begin{equation*}
        a_n - b_n = 0.
      \end{equation*}
      Therefore $\lim (a_n - b_n) = 0$, yet $\lim a_n \neq \lim b_n$.
    \end{proof}
    \vspace{.25in}

    \item If $(b_n) \to b$, then $|b_n| \to |b|$\\
    
    \begin{proof}
      Suppose $(b_n) \to b$. Consider that through the triangle inequality we know that (Exercise 1.2.6d),
      \begin{equation*}
        |b_n - b| \geq ||b_n| - |b||. 
      \end{equation*}
      Since $(b_n) \to b$ we know that for all $\epsilon > 0$ there exists an $N \in \Nats $ such that for all $n\geq N$,
      \begin{equation*}
        |b_n - b|< \epsilon.
      \end{equation*}
      Thus it follows simply that,
      \begin{equation*}
        ||b_n| - |b|| < \epsilon.
      \end{equation*}
      Thus we have shown that $|b_n| \to |b|$. 
    \end{proof}

    \item If $(a_n) \to a$ and $(b_n - a_n) \to 0$, then $(b_n) \to a$.\\
    
    \begin{proof}
      Suppose $(a_n) \to a$ and $(b_n - a_n) \to 0$. Rewriting the expression $|b_n - a|$,
      \begin{equation*}
        |b_n - a| = |b_n - a_n + a_n - a|.
      \end{equation*}
      By the triangle inequality,
      \begin{equation*}
        |b_n - a_n + a_n - a| \le |b_n - a_n| + |a_n - a|.
      \end{equation*}
      Since $(a_n) \to a$ and $(b_n - a_n) \to 0$ we know that for all $\epsilon > 0$ there exists a $N \in \Nats$ such that for all $n \geq N$,
      \begin{equation*}
        |a_n - a| < \dfrac{\epsilon}{2},
      \end{equation*} 
      \begin{equation*}
        |a_n - b_n| < \dfrac{\epsilon}{2}.
      \end{equation*}
      Therefore,
      \begin{align*}
        |b_n - a| &\le |b_n - a_n| + |a_n - a|,\\
        &< \dfrac{\epsilon}{2} + \dfrac{\epsilon}{2},\\
        <\epsilon.
      \end{align*}
      Thus we have shown that, $(b_n) \to a$.
    \end{proof}
    \vspace{.25in}



    \item If $a_n \to 0$ and $|b_n - b| \le a_n$ for all $n \in \Nats$ then  $(b_n) \to b$.\\
    
    \begin{proof}
      Suppose $a_n \to 0$ and $|b_n - b| \le a_n$ for all $n \in \Nats$. Since $a_n \to 0$ we know that for all $\epsilon > 0$
      there exists an $N \in \Nats$ such that for all $n \geq N$, 
      \begin{equation*}
        |a_n - 0| = |a_n| < \epsilon.
      \end{equation*}
      Therefore we chain these inequalities and get,
      \begin{equation*}
        |b_n - b| \le |a_n| < \epsilon
      \end{equation*}
      Thus we have shown that, $(b_n) \to b$. 
    \end{proof}
    
  \end{enumerate}

\end{exercise}
\vspace{.5in}






\begin{exercise}{Supplemental 3}
Show that if $|b_n|\to 0$, then $b_n\to 0$.
Then show that this statement is false if we replace $0$
with any other real number.\\

\begin{proof}
  Suppose the sequence $|b_n|\to 0$. Since $|b_n|$ converges we know that for all $\epsilon > 0$ there exists an $n \in \Nats$
  where for all $n \geq N$,
  \begin{equation*}
    ||b_n| - 0| \le \epsilon.
  \end{equation*}
  Rewriting the expression,
  \begin{equation*}
    ||b_n| - 0| = ||b_n|| = |b_n| = |b_n - 0|.
  \end{equation*}
  Therefore the following inequality still holds,
  \begin{equation*}
    |b_n - 0| \le \epsilon.
  \end{equation*}
Thus we have shown that, $b_n\to 0$.\\

Suppose we where to replace $0$ with a $1$ and consider the sequence,
\begin{equation*}
  b_n = (-1)^n.
\end{equation*}
Clearly $|b_n|\to 1$ however $b_n\to -1$ thus the statement does not hold.
\end{proof}
\end{exercise}
\vspace{.5in}






\begin{exercise}{Supplemental 4}
  
Consider the series $\sum_{n=1}^\infty 1/n^2$. 
Give a careful proof by induction that the partial sums
\[
s_k = \sum_{n=1}^k 1/n^2
\]
satisfy $s_k \le 2 - 1/k$.\\

\begin{proof}
  Consider the case where $k = 1$, 
  \begin{equation*}
    s_1 =\dfrac{1}{1^2} = 1.
  \end{equation*} 
  Clearly,
  \begin{equation*}
    s_1 = 1 \le 2- \dfrac{1}{1} = 1.
  \end{equation*}
We will now proceed by induction on $k$. Suppose there exists some $k \in \Nats$ such that,
\begin{equation*}
  s_k \le 2 - 1/k
\end{equation*}
Note that by the definition of $s_k$ we know that,
\begin{equation*}
  s_{k+1} = s_k + \dfrac{1}{(k+1)^2}.
\end{equation*}
From our induction hypothesis and using the same algebraic argument as example 2.4.4. we get that,
\begin{align*}
  s_{k+1} &= s_k + \dfrac{1}{(k+1)^2}\\
  &\le 2 - \dfrac{1}{k} + \dfrac{1}{(k+1)^2},\\
  &< 2 - \dfrac{1}{k} + \dfrac{1}{(k)(k+1)},\\
  &=  2 - \dfrac{1}{k} + (\dfrac{1}{(k)} - \dfrac{1}{(k+1)}),\\
  &=  2 - \dfrac{1}{(k+1)}.
\end{align*}
Thus we have proven through induction that the partial sums $s_k$ satisfy $s_k \le 2 - 1/k$.
\end{proof}
\end{exercise}
\vspace{.5in}







\begin{exercise}{2.4.3(a)} Show that the following sequence converges and find the limit,
  \begin{equation*}
    \sqrt{2}, \sqrt{2+\sqrt{2}}, \sqrt{2+ \sqrt{2+\sqrt{2}}} ,\dots
  \end{equation*}
  
  \begin{proof}
    First we will prove that the sequence is bounded above by $2$ using induction. Note that the sequence $a_n$ written in the form of a recurrence relation,
    \begin{equation}
      a_{n+1} = \sqrt{2+ a_n}
    \end{equation}
    Note that when $n = 1$ we see that,
    \begin{equation}
      a_1 = \sqrt{2} < 2.
    \end{equation}
    now suppose that for some $n \in \Nats$ the following is true,
    \begin{equation*}
      a_n \le 2.
    \end{equation*}
    Consider the term $a_{n+1}$ by the definition,
    \begin{align*}
      a_{n+1} &= \sqrt{2 + a_n},\\
      &\le \sqrt{2+2},\\
      &\le 2.      
    \end{align*}
    Therefore by induction for all $n \in \Nats$ we have shown that $a_n \le 2$ and thus the sequence $a_n$ is bounded above by 2.\\

    Now we will prove that the sequence is monotone increasing through induction. First note that,
    \begin{equation*}
      a_2 = \sqrt{2+ \sqrt{2}} \geq \sqrt{2} = a_1
    \end{equation*}
    Now suppose that for some $n \in \Nats$,
    \begin{equation*}
      a_n \geq a_{n-1}.
    \end{equation*}
    Consider the term $a_{n+1}$ by the definition,
    \begin{equation*}
      a_{n+1} = \sqrt{2+ a_n} \geq \sqrt{2+ a_{n-1}} = a_n.
    \end{equation*}
    Thus we have shown that for all $n \in \Nats$ that $a_n \geq a_{n-1}$.\\

    By the Monotone convergence theorem we can be certain that the series converges. To find where it converges consider the fizxed point equation,
    \begin{equation*}
      \phi(x) = \sqrt{2 + x}.
    \end{equation*}
    Finding the fixed points for $\phi$,
    \begin{align*}
      x &= \sqrt{2+ x},\\
      x^2 &= 2 + x,\\
      x^2 - x - 2  &= 0,\\
      (x-2)(x+1) &= 0.
    \end{align*}
Since the sequence only produces positive real numbers we know that the series must converge to a value of 2.
  \end{proof}
\end{exercise}


\end{document}