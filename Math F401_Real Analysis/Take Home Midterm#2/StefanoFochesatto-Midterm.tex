%%%%%%%%%%%%%%%%%%%%%%%%%%%%%%%%%%%%%%%%%%%%%%%%%%%%%%%%%%%%%%%%%%%%%%%%%%%%%%%%%%%%%%%
%%%%%%%%%%%%%%%%%%%%%%%%%%%%%%%%%%%%%%%%%%%%%%%%%%%%%%%%%%%%%%%%%%%%%%%%%%%%%%%%%%%%%%%
% 
% This top part of the document is called the 'preamble'.  Modify it with caution!
%
% The real document starts below where it says 'The main document starts here'.

\documentclass[12pt]{article}

\usepackage{amssymb,amsmath,amsthm}
\usepackage[top=1in, bottom=1in, left=1.25in, right=1.25in]{geometry}
\usepackage{fancyhdr}
\usepackage{enumerate}

% Comment the following line to use TeX's default font of Computer Modern.
\usepackage{times,txfonts}

\newtheoremstyle{homework}% name of the style to be used
  {18pt}% measure of space to leave above the theorem. E.g.: 3pt
  {12pt}% measure of space to leave below the theorem. E.g.: 3pt
  {}% name of font to use in the body of the theorem
  {}% measure of space to indent
  {\bfseries}% name of head font
  {:}% punctuation between head and body
  {2ex}% space after theorem head; " " = normal interword space
  {}% Manually specify head
\theoremstyle{homework} 

% Set up an Exercise environment and a Solution label.
\newtheorem*{exercisecore}{Exercise \@currentlabel}
\newenvironment{exercise}[1]
{\def\@currentlabel{#1}\exercisecore}
{\endexercisecore}

\newcommand{\localhead}[1]{\par\smallskip\noindent\textbf{#1}\nobreak\\}%
\newcommand\solution{\localhead{Solution:}}

%%%%%%%%%%%%%%%%%%%%%%%%%%%%%%%%%%%%%%%%%%%%%%%%%%%%%%%%%%%%%%%%%%%%%%%%
%
% Stuff for getting the name/document date/title across the header
\makeatletter
\RequirePackage{fancyhdr}
\pagestyle{fancy}
\fancyfoot[C]{\ifnum \value{page} > 1\relax\thepage\fi}
\fancyhead[L]{\ifx\@doclabel\@empty\else\@doclabel\fi}
\fancyhead[C]{\ifx\@docdate\@empty\else\@docdate\fi}
\fancyhead[R]{\ifx\@docauthor\@empty\else\@docauthor\fi}
\headheight 15pt

\def\doclabel#1{\gdef\@doclabel{#1}}
\doclabel{Use {\tt\textbackslash doclabel\{MY LABEL\}}.}
\def\docdate#1{\gdef\@docdate{#1}}
\docdate{Use {\tt\textbackslash docdate\{MY DATE\}}.}
\def\docauthor#1{\gdef\@docauthor{#1}}
\docauthor{Use {\tt\textbackslash docauthor\{MY NAME\}}.}
\makeatother

% Shortcuts for blackboard bold number sets (reals, integers, etc.)
\newcommand{\Reals}{\ensuremath{\mathbb R}}
\newcommand{\Nats}{\ensuremath{\mathbb N}}
\newcommand{\Ints}{\ensuremath{\mathbb Z}}
\newcommand{\Rats}{\ensuremath{\mathbb Q}}
\newcommand{\Cplx}{\ensuremath{\mathbb C}}
%% Some equivalents that some people may prefer.
\let\RR\Reals
\let\NN\Nats
\let\II\Ints
\let\CC\Cplx

%%%%%%%%%%%%%%%%%%%%%%%%%%%%%%%%%%%%%%%%%%%%%%%%%%%%%%%%%%%%%%%%%%%%%%%%%%%%%%%%%%%%%%%
%%%%%%%%%%%%%%%%%%%%%%%%%%%%%%%%%%%%%%%%%%%%%%%%%%%%%%%%%%%%%%%%%%%%%%%%%%%%%%%%%%%%%%%
% 
% The main document start here.

% The following commands set up the material that appears in the header.
\doclabel{Math 401: Take Home Midterm \#2}
\docauthor{Stefano Fochesatto}
\docdate{\today}

\begin{document}
%NOT GOOD PROOF
%
%
%

\begin{exercise}{1} Suppose $f: A \to \Reals$ and $c$ is a limit point of $A$. Suppose $f(x) \geq 0$ for all 
  $x \in X$ and that $\lim_{x \to c}f(x)$ exist. Show that the limit is non-negative. Provide two proofs, one $\epsilon-\delta$
  style, and the other using the sequential characterization of limits\\


  \begin{proof}
    Suppose $f: A \to \Reals$ and $c$ is a limit point of $A$. Suppose $f(x) \geq 0$ for all 
  $x \in X$ and that $\lim_{x \to c}f(x) = L$. Also suppose for the sake of contradiction that $L < 0$. By the definition of a Functional Limit
  we know that for all $\epsilon > 0$ there exists a $\delta > 0$ such that whenever $0 < |x - c|< \delta$ it follows that $|f(x) - L|< \epsilon$.
  Now consider $\epsilon = -L$. By substitution we get the following inequality,
  \begin{equation*}
    |f(x) - L|< -L.
  \end{equation*}
  Expanding the inequality and solving for $f(x)$ we find,
  \begin{align*}
    L < &f(x) - L < -L,\\
    2L < &f(x) < 0.
  \end{align*}
  Since $2L < 0$ we find that our final inequality implies that $f(x)<0$ and thus a contradiction. 
  
\end{proof}

\begin{proof}
  Suppose $f: A \to \Reals$ and $c$ is a limit point of $A$. Suppose $f(x) \geq 0$ for all 
  $x \in X$ and that $\lim_{x \to c}f(x) = L$.  Sequential Characterization of Limits we know that if $\lim_{x \to c}f(x) = L$ then for all sequences 
  $(x_n)\subseteq X$ satisfying $(x_n) \to c$ it we know that $f(x_n) \to L$. By the Order Limit Theorem we know that if $f(x_n)\geq 0$ then $L \geq 0$.
\end{proof}

\end{exercise}
\vspace{.5in}





\begin{exercise}{2} Let $a_n$ be a sequence of numbers such that for some $M \in \Reals$, $\sum_{n = 1}^{\infty}a_nM^n$ converges. Suppose
  that $|x| < M$. Show that $\sum_{n = 1}^{\infty}a_nx^n$ converges absolutely. Give an example to show that divergence is possible if $|x| = |M|$.
  Hint: $(a_nM^n)$ converges to zero and hence bounded. \\
  \begin{proof}
    Suppose that $a_n$ be a sequence of numbers such that for some $M \in \Reals$, $\sum_{n = 1}^{\infty}a_nM^n$ converges and that for some $x \in \Reals$
    $|x|<M$. Consider the series $\sum_{n = 1}^{\infty}|a_nx^n|$, and through algebra we see that,
    \begin{align*}
      \sum_{n = 1}^{\infty}|a_nx^n| &= \sum_{n = 1}^{\infty}|a_n||x|^n,\\
       &= \sum_{n = 1}^{\infty}|a_n|M^{n}M^{-n}|x|^n,\\
       &= \sum_{n = 1}^{\infty}|a_n|M^{n} \left(\dfrac{|x|}{M}\right)^n.
    \end{align*}
   Since the sequence $(a_nM^n)$ converges to zero its bounded, therefore there exists some $||a_n|M^{n}|<A$. Therefore we get the following inequality,
   \begin{equation*}
    \sum_{n = 1}^{\infty}|a_nx^n| \le \sum_{n = 1}^{\infty} A \left(\dfrac{|x|}{M}\right)^n.
   \end{equation*}
   Recall that $|x|<M$ and therefore we can surmise that $|\frac{|x|}{M}|<1$. Thus it follows that,
   \begin{equation*}
    \sum_{n = 1}^{\infty} A \left(\dfrac{|x|}{M}\right)^n = \dfrac{A}{1 - \dfrac{|x|}{M}},
   \end{equation*}
   is a convergent geometric series. Thus by the Comparison Test we get that, $\sum_{n = 1}^{\infty}|a_nx^n|$ is convergent and therefore $\sum_{n = 1}^{\infty}a_nx^n$ converges absolutely.
  \end{proof}
\vspace{.25in}

\solution  To show that divergence is possible if $|x| = |M|$, let $M = -1$, $x = 1$ and $a_n = \dfrac{1}{n}$. By substitution this gives us the following,
\begin{equation*}
  \sum_{n = 1}^{\infty}a_nM^n = \sum_{n = 1}^{\infty}\dfrac{(-1)^n}{n}.
\end{equation*}
A convergent alternating series. Substituting $x$ we get,
\begin{equation*}
  \sum_{n = 1}^{\infty}a_nx^n = \sum_{n = 1}^{\infty}\dfrac{1}{n}.
\end{equation*}
The famously divergent harmonic series. 
\end{exercise}
\vspace{.5in}



\begin{exercise}{3} Suppose that $f: (0,1]\to \Reals$ is uniformly continuous. Show that $\lim_{x\to 0} f(x)$ exists. 

  \begin{proof}

    Suppose that $f: (0,1]\to \Reals$ is uniformly continuous. Consider some $x_n \subseteq (0,1]$ such that $x_n \to c$. Since $x_n$ is convergent, by 
    Theorem 2.6.4 it is also a cauchy sequence. Recall that in Exercise 4.4.6(b) we proved that on a uniformly continuous function, if $x_n$ is cauchy then $f(x_n)$ is also cauchy.
    By Theorem 2.6.4 we know that since $f(x_n)$ is cauchy it also converges, and thus there exists some $L$ such that $f(x_n) \to L$.\\

    Demonstrating that $f(x_n) \to L$ is the same for all sequences $x_n$, we first suppose $x_n \to c$ and $z_n \to c$ such that $f(x_n)\to L_x$ and $f(z_n)\to L_z$
    where $L_x \neq L_z$. By the Algebraic Limit Theorem we know that $|x_n - z_n| \to 0$ and we also know that since $L_x \neq L_z$ there must exists some $\epsilon_0$ that has the following property,
    \begin{equation*}
      |f(x_n) - f(z_n)|\geq \epsilon_0.
    \end{equation*}
    Thus by Theorem 4.4.5 $f$ is not uniformly continuous and thus a contradiction. Therefore by Theorem 4.2.3 (Sequential Criterion for Functional Limits) we know that  $\lim_{x\to 0} f(x)$ exists. 
  
  \end{proof}


\end{exercise}
\vspace{.5in}



\begin{exercise}{4} Let $f$ be a function defined on all of $\Reals$, and assume there is a constant $c$ such that 
  $0<c<1$ and for all $x,y \in \Reals$,
  \begin{equation*}
    |f(x) - f(y)|\le c|x - y|
  \end{equation*}
  \begin{enumerate}
    \item Show that $f$ is continuous on $\Reals$.\\
    \begin{proof}
      Let $\epsilon > 0$. Now consider $\delta = \frac{\epsilon}{c}$ then for all $|x - y| < \delta$ we see that from the inequality above we get that,
      \begin{align*}
    |f(x) - f(y)|&\le c|x - y|,\\
    \dfrac{1}{c}|f(x) - f(y)|&\le |x - y|,\\
    \dfrac{1}{c}|f(x) - f(y)|&< \delta\\
    \dfrac{1}{c}|f(x) - f(y)|&< \dfrac{\epsilon}{c},\\
    |f(x) - f(y)|&< \epsilon.\\
      \end{align*}
      Thus by definition $f$ is continuous on $\Reals$.
    \end{proof}
    \vspace{.25in}


    \item Pick some point $y_1 \in \Reals$ and construct the sequence,
    \begin{equation*}
      (y_1,f(y_1),f(f(y_1)),\dots f^n(y_1)).
    \end{equation*}
    In general, if $y_{n+1} = f(y_n)$, show that the resulting sequence $(y_n)$ is a cauchy sequence. Hence we may let $y = \lim y_n$.\\

    \begin{proof}
      Suppose the the sequence above and note that by the previous inequality for any two elements in the sequence $f^n(y_1), f^m(y_1)$ where $n > m$ we get,
      \begin{equation*}
        |f^n(y_1) - f^m(y_1)|\le c|f^{n-1}(y_1) - f^{m-1}(y_1)|.
      \end{equation*}
      Continually applying the previous inequality to the right hand side, we get an upper bound for $|f^n(y_1) - f^m(y_1)|$,
      \begin{align*}
        |f^n(y_1) - f^m(y_1)|&\le c|f^{n-1}(y_1) - f^{m-1}(y_1)|\\
        &\le c^2|f^{n-2}(y_1) - f^{m-2}(y_1)|\\
        &\le c^m|f^{n-m}(y_1) - f^{m-m}(y_1)|\\
        &= c^m|f^{n-m}(y_1) - y_1|.
      \end{align*}
      Let $M = |f^{n-m}(y_1) - y_1|$, and note that since $0<c<1$ by Example 2.5.3 we know that $c^mM \to 0$. By the Order Limit Theorem we know that 
      $|f^n(y_1) - f^m(y_1)|$ is convergent and therefore $(y_n)$ is a cauchy sequence.
    \end{proof}
    \vspace{.25in}


    \item Prove that $y$ is a fixed point of $f$ (i.e., $f(y) = y$) and that it is unique in this regard. \\
    \begin{proof}
      By the preceding problem we have shown that $y = \lim y_n$ where $y_{n} = f^n(y_1)$. Now consider $f(y)$, and by substitution we get that,
      \begin{align*}
        f(y) &= f(\lim(y_n)),\\
         &= f(\lim(y_n)),\\
         &= f(\lim f^n(y_1)),\\
         &= \lim f^{n+1}(y_1)),\\
         &= y.
      \end{align*}
      Thus we have shown that y is fixed. Now suppose there exists some $x$ with the property that $f(x) = x$. By substitution into our initial inequality we get,
      \begin{align*}
        |f(x) - f(y)| \le c|x - y|,\\
        |x - y| \le c|x - y|,\\
        1 \le c.
      \end{align*}
      Thus a contradiction, therefore it must be the case that $y$ is unique.
    \end{proof}
    \vspace{.25in}

    \item Finally prove that if $x$ is any arbitrary point in $\Reals$, then the sequence $(x,f(x), f(f(x)),\dots)$ converges to the $y$ defined in part b.\\
    \begin{proof}
      Suppose some $x \in \Reals$ and $y$ with the property that $y = f(y)$. By substitution we get the following inequality,
      \begin{equation*}
        |f(x) - y| \le c|x - y|.
      \end{equation*}
      Similarly to part $b$ applying this inequality to our sequence yields an upper bound,
      \begin{align*}
        |f^n(x) - y| &\le c|f^{n_1}(x) - y|\\
         &\le c^n|x - y|.
      \end{align*}
      Since $0<c<1$ by Example 2.5.3 we know that $c^m|x - y| \to 0$. By the Order Limit Theorem and Absolute Convergence we know that $f^n(x) - y \to 0$.
      Finally by the Algebraic Limit Theorem we know that $\lim f^n(x) \to y$.
    
    \end{proof}
  \end{enumerate}
\end{exercise}
\vspace{.5in}






\begin{exercise}{5} Suppose that $f: (0,1) \to \Reals$ is continuous and that $\lim_{x \to 0}f(x) = \infty$ and $\lim_{x \to 1}f(x) = \infty$. Show that $f$ obtains a minimum on $(0,1)$.\\

  \begin{proof} 
    Suppose that $f: (0,1) \to \Reals$ is continuous and that $\lim_{x \to 0}f(x) = \infty$ and $\lim_{x \to 1}f(x) = \infty$. By the definition of infinite limit, for all $M>0$ we can find a $\delta_0, \delta_1 > 0$
    such that whenever $0< |x - 0| < \delta_0$ and $0< |x - 1| < \delta_1$ it follows that $f(x)>M$. let $a = 0 +\delta_0$ and $b = 1 - \delta_1$ now consider the closed interval $[a , b]$.
    Note that $[a , b] \subseteq (0,1)$. By Example 3.2.9(ii) we know that $[a,b]$ is closed and by definition its bounded above by $1$ and below by $0$, thus $[a,b]$ is a compact set. By Theorem 4.4.1 we can conclude that $f$ is continuous on $[a,b]$.  
    By the Extreme Value Theorem there exists some $x_0 \in [a , b]$ such that $f(x_0) \le f(x)$. Note that $x_0 \in (0,1)$ and thus $f$ obtains a minimum on $(0,1)$
  \end{proof}
  \end{exercise}
\vspace{.5in}




\begin{exercise}{6} Show that if $f:[a,b] \to \Reals$ is strictly increasing and continuous, then it has a continuous inverse function $f^{-1}:[f(a),f(b)] \to [a,b]$. Use this result to show that $x^{1/n}$ is continuous for each $n\in \Nats$.\\

\begin{proof}
  Suppose $f:[a,b] \to \Reals$ is strictly increasing and continuous. Recall that to prove a function $f$ has an inverse we must demonstrate
  that $f$ is a bijection.\\
  
  Suppose $x,y \in [a,b]$ such that $x \neq y$. Without loss of generality lets suppose that $x > y$. Since $f$ is a strictly increasing function if 
  $x > y$ then it follows that $f(x) > f(y)$ and therefore $f(x) \neq f(y)$. Thus $f$ is an injection\\

  Suppose $y \in  [f(a),f(b)]$. By definition we know that $f(a) \le y \le f(b)$. Since $f$ is continuous we know that by the Intermediate Value Theorem there exists some 
  $x \in [a,b]$ where $f(x) = y$. Thus $f$ is surjective on $f:[a,b] \to [f(a),f(b)]$. Since $f$ is a bijection we know there exists an inverse function $f^{-1}:[f(a),f(b)] \to [a,b]$.\\

 Recall that $f$ is continuous and by definition for any $c \in [a,b]$, for all $\epsilon_0 >0$ there exists a $\delta_0> 0$ such that whenever $|x - c|<\delta_0$ it follows that $|f(x) - f(c)|< \epsilon_0$.
 Now consider $f^{1}$ and let $z \in [f(a),f(b)]$ with the property that $f(c) = z$. Consider $\delta = \epsilon_0$ then for all $y \in [f(a),f(b)]$ with the property that $f(x) = y$, $|y - z|< \delta$ implies,
  \begin{align*}
   |f^{-1}(y) - f^{-1}(z)| &= |f^{-1}(f(x)) - f^{-1}(f(c))|,\\
   &= |x - c|,\\
   &< \delta = \epsilon_0.
   \end{align*} 
Thus $f^{-1}$ is continuous. \\

Using this result to show that $f(x) = x^{1/n}$ is continuous for each $n in \Nats$. Consider the strictly increasing and continuous function $g(x) = x^n$, and note that $f \circ g = (x^n)^{1/n} = x$ thus by 
our previous result  $f(x) = x^{1/n}$ is continuous.

\end{proof}
\end{exercise}
\vspace{.5in}








\begin{exercise}{7} Suppose $f: [0,1] \to \Reals$ is continuous and that $f([0,1]) \subseteq (0,1)$. Prove that there is a solution of the equation $f(x) = x$. 
  
\begin{proof}
  Suppose $f: [0,1] \to \Reals$ is continuous and that $f([0,1]) \subseteq (0,1)$. Consider a continuous function $g:[0,1] \to \Reals$
  defined by $g(x) = f(x) - x$. Clearly when $g(x) = 0$ we have a solution for $f(x) = x$. Consider $g(0) = f(0) - 0$, and suppose $g(0) \neq 0$ (otherwise we would have a solution). Then since $f(0) \in f([0,1])$ which is a 
  subset of $(0,1)$ it must be the case that $g(0) > 0$. Similarly consider $g(1) = f(1) - 1$, and suppose $g(1) \neq 1$. Since $f(0) \in f([0,1])$ which is a 
  subset of $(0,1)$ it follows that $g(1) < 1$. By The Intermediate Value Theorem that since $g$ is continuous and $g(1)<0<g(0)$ there must exist a point $c \in (a,b)$ such that $g(c) = 0$.
  
\end{proof}
\end{exercise}
\vspace{.5in}






\begin{exercise}{8} If $f: [a,b] \to \Reals$ is one-to-one, then there exists an inverse function $f^{-1}$ defined on the range of $f$ given by $f^{-1}(y) = x$
  where $y = f(x)$. In Exercise 4.5.8 we saw that if $f$ is continuous on $[a,b]$ then $f^{-1}$ is continuous on its domain. Let's add the assumption that $f$ is differentiable 
  on $[a,b]$ with $f'\neq 0$ for all $x \in [a,b]$. Show that $f^{-1}$ is differentiable with
  \begin{equation*}
    (f^{-1})'(y) = \dfrac{1}{f'(x)}, \text{ where $y = f(x)$}
  \end{equation*}


  \begin{proof} Suppose  $f: [a,b] \to \Reals$ is one-to-one, continuous, differentiable function with an inverse function $f^{-1}$ defined on $f^{-1}: [f(a),f(b)] \to [a,b]$ and the property that $f'\neq 0$.    
    Let $f(c) \in [f(a),f(b)]$ and consider $f(y_n) \subseteq [f(a),f(b)]$ such that $f(y_n) \to f(c)$. By the definition of the derivative using the sequential characterization of a limit,
    \begin{equation*}
      f'^{-1}(f(c)) = \lim_{n \to \infty} \dfrac{f'^{-1}(f(y_n)) - f'^{-1}(f(c))}{f(y_n) - f(c)}.
    \end{equation*}
    Simplifying to get our limit in terms of $f$ rather than $f^{-1}$ and solving using the algebraic limit theorem,
    \begin{align*}
      f'^{-1}(f(c)) &= \lim_{n \to \infty} \dfrac{f'^{-1}(f(y_n)) - f'^{-1}(f(c))}{f(y_n) - f(c)},\\
       &= \lim_{n \to \infty} \dfrac{y_n - c}{f(y_n) - f(c)},\\
       &= \lim_{n \to \infty} \left(\dfrac{f(y_n) - f(c)}{y_n - c}\right)^{-1},\\
       &= \dfrac{1}{f'(c)}.
    \end{align*}



  \end{proof}



\end{exercise}











\end{document}