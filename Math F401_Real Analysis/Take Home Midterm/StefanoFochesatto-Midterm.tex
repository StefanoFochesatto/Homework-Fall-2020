%%%%%%%%%%%%%%%%%%%%%%%%%%%%%%%%%%%%%%%%%%%%%%%%%%%%%%%%%%%%%%%%%%%%%%%%%%%%%%%%%%%%%%%
%%%%%%%%%%%%%%%%%%%%%%%%%%%%%%%%%%%%%%%%%%%%%%%%%%%%%%%%%%%%%%%%%%%%%%%%%%%%%%%%%%%%%%%
% 
% This top part of the document is called the 'preamble'.  Modify it with caution!
%
% The real document starts below where it says 'The main document starts here'.

\documentclass[12pt]{article}

\usepackage{amssymb,amsmath,amsthm}
\usepackage[top=1in, bottom=1in, left=1.25in, right=1.25in]{geometry}
\usepackage{fancyhdr}
\usepackage{enumerate}

% Comment the following line to use TeX's default font of Computer Modern.
\usepackage{times,txfonts}

\newtheoremstyle{homework}% name of the style to be used
  {18pt}% measure of space to leave above the theorem. E.g.: 3pt
  {12pt}% measure of space to leave below the theorem. E.g.: 3pt
  {}% name of font to use in the body of the theorem
  {}% measure of space to indent
  {\bfseries}% name of head font
  {:}% punctuation between head and body
  {2ex}% space after theorem head; " " = normal interword space
  {}% Manually specify head
\theoremstyle{homework} 

% Set up an Exercise environment and a Solution label.
\newtheorem*{exercisecore}{Exercise \@currentlabel}
\newenvironment{exercise}[1]
{\def\@currentlabel{#1}\exercisecore}
{\endexercisecore}

\newcommand{\localhead}[1]{\par\smallskip\noindent\textbf{#1}\nobreak\\}%
\newcommand\solution{\localhead{Solution:}}

%%%%%%%%%%%%%%%%%%%%%%%%%%%%%%%%%%%%%%%%%%%%%%%%%%%%%%%%%%%%%%%%%%%%%%%%
%
% Stuff for getting the name/document date/title across the header
\makeatletter
\RequirePackage{fancyhdr}
\pagestyle{fancy}
\fancyfoot[C]{\ifnum \value{page} > 1\relax\thepage\fi}
\fancyhead[L]{\ifx\@doclabel\@empty\else\@doclabel\fi}
\fancyhead[C]{\ifx\@docdate\@empty\else\@docdate\fi}
\fancyhead[R]{\ifx\@docauthor\@empty\else\@docauthor\fi}
\headheight 15pt

\def\doclabel#1{\gdef\@doclabel{#1}}
\doclabel{Use {\tt\textbackslash doclabel\{MY LABEL\}}.}
\def\docdate#1{\gdef\@docdate{#1}}
\docdate{Use {\tt\textbackslash docdate\{MY DATE\}}.}
\def\docauthor#1{\gdef\@docauthor{#1}}
\docauthor{Use {\tt\textbackslash docauthor\{MY NAME\}}.}
\makeatother

% Shortcuts for blackboard bold number sets (reals, integers, etc.)
\newcommand{\Reals}{\ensuremath{\mathbb R}}
\newcommand{\Nats}{\ensuremath{\mathbb N}}
\newcommand{\Ints}{\ensuremath{\mathbb Z}}
\newcommand{\Rats}{\ensuremath{\mathbb Q}}
\newcommand{\Cplx}{\ensuremath{\mathbb C}}
%% Some equivalents that some people may prefer.
\let\RR\Reals
\let\NN\Nats
\let\II\Ints
\let\CC\Cplx

%%%%%%%%%%%%%%%%%%%%%%%%%%%%%%%%%%%%%%%%%%%%%%%%%%%%%%%%%%%%%%%%%%%%%%%%%%%%%%%%%%%%%%%
%%%%%%%%%%%%%%%%%%%%%%%%%%%%%%%%%%%%%%%%%%%%%%%%%%%%%%%%%%%%%%%%%%%%%%%%%%%%%%%%%%%%%%%
% 
% The main document start here.

% The following commands set up the material that appears in the header.
\doclabel{Math 401: Take Home Midterm}
\docauthor{Stefano Fochesatto}
\docdate{\today}

\begin{document}

\begin{exercise}{1}Let $A$ and $B$ be non empty sets that are bounded above.Suppose $\sup A < \sup B$. Prove that there is an element in $B$ that is an
	upper bound for $A$.\\

	\begin{proof}
		Suppose that $A$ and $B$ be non empty sets that are bounded above and that $\sup A < \sup B$. Let $x = \sup A$ and $y = \sup B$. Now consider some $z$
		such that, $0 < z < y-x$. Through some algebra we can see that, $x < y - z$  and therefore the term $y - z$ must be an upper bound for $A$ since it is larger than its least upper bound. Also 
		note that $y - z < y$ and therefore $y-z$ must be contained in $B$.
	\end{proof}

\end{exercise}


\begin{exercise}{2} In class we proved that $\Nats^2$ is countably infinite. Use this fact and a proof by induction to show that $\Nats^n$ is countably infinite for every $n \in \Nats$. \\

  \begin{proof}
    Consider the base case where $n = 1$, clearly $\Nats^1$ is countably infinite and we have proven that $\Nats^2$ is countably infinite. We will proceed by induction on $n$. 
    Suppose there exists some $n\in \Nats$ such that $\Nats^n$ is countably infinite. By the induction hypothesis there exists some bijection 
     $g: \Nats^n \to \Nats$. Now consider the bijection we proved in class $f: \Nats^2 \to \Nats$. Note that the composition of these two functions gives us, 
     $f\circ g: \Nats^{n+1} \to \Nats $ and since $f\circ g$ is a composition of bijections it must also be a bijection. Thus by induction we have shown that for all $n\in \Nats$
     $\Nats^n$ is countably infinite.     
  \end{proof}
  
\end{exercise}
\vspace{.5in}



\begin{exercise} {3} Compute,
  \begin{equation*}
    \lim_{n \to \infty}\dfrac{3^n}{n!}.
  \end{equation*}
  A fully rigorous proof will involve a proof by induction. \\

\begin{proof}
  We will proceed by induction to prove the following inequality for all $n in \Nats$ when $n \geq 9$,
  \begin{equation*}
    \dfrac{1}{n!} \le \dfrac{1}{4^n}.
  \end{equation*}
  Consider the base case where $n = 9$,
  \begin{equation*}
    \dfrac{1}{4^n} = \dfrac{1}{4^9} \le \dfrac{1}{9!} = \dfrac{1}{n!}.
  \end{equation*}
  Now suppose there exists some $n \in \Nats$ where $n \geq 9$ such that,
  \begin{equation*}
    \dfrac{1}{n!} \le \dfrac{1}{4^n}.
  \end{equation*}
  Now consider $\frac{1}{n!}$, and by substituting our induction hypothesis,
  \begin{align*}
    \dfrac{1}{n+1!}  &= \dfrac{1}{n+1}\dfrac{1}{n!}\\
    &\le  \dfrac{1}{n+1}\dfrac{1}{4^n}\\
    &\le  \dfrac{1}{4}\dfrac{1}{4^n}\\
    &\le  \dfrac{1}{4^{n+1}}.
  \end{align*}
  Therefore for large $n$ we know that, 
  \begin{equation*}
  \dfrac{3^n}{n!} \le \dfrac{3^n}{4^n} = \dfrac{3}{4}^n.
  \end{equation*}
  Furthermore it must also be the case that, 
  \begin{equation*}
    \lim_{n \to \infty}\dfrac{3^n}{n!} \le  \lim_{n \to \infty}\dfrac{3}{4}^n.
  \end{equation*}
  Clearly, $\dfrac{3^n}{n!}$ is bounded below by $0$ and since,
  \begin{equation*}
    \lim_{n \to \infty}\dfrac{3}{4}^n \to 0.
  \end{equation*}
It must be the case that,
\begin{equation*}
  \lim_{n \to \infty}\dfrac{3^n}{n!} = 0.
\end{equation*}

\end{proof}

\end{exercise}
\vspace{.5in}



\begin{exercise} {4} Let $(x_n)$ and $(y_n)$ be given, and define $(z_n)$ to be the "shuffled"
  sequence $(x_1,y_1,x_2,y_2,\dots,x_n,y_n,\dots)$. Prove that $(z_n)$ is convergent if and only if 
  $(x_n)$ and $(y_n)$ are both convergent with $\lim x_n = \lim y_n$.\\

\begin{proof}
  Suppose convergent sequences $(x_n)$ and $(y_n)$ such that $\lim x_n = \lim y_n = l$. consider a sequence $(z_n)$ such that
  $(x_1,y_1,x_2,y_2,\dots,x_n,y_n,\dots)$. Note that $(z_{2n}) = (y_n)$ and $(z_{2n-1}) = (x_n)$. Let $\epsilon > 0$. Since 
  $(x_n)$ and $(y_n)$ converge we know that there exists $N_x,N_y \in \Nats$ such that,
  \begin{equation*}
    |x_n - l|<\epsilon,
  \end{equation*}
  \begin{equation*}
    |y_n - l|<\epsilon.
  \end{equation*}
  Consider an $N \in \Nats$ such that $N = \max(N_x,N_m)$. By substitution we get that for all odd and even values of $(z_n)$ we get,
  \begin{equation*}
    |z_n - l| < \epsilon.
  \end{equation*}
Thus $(z_n)$ is convergent. 
\end{proof}
\vspace{.25 in}

\begin{proof}
  Suppose a sequence $(z_n) =(x_1,y_1,x_2,y_2,\dots,x_n,y_n,\dots)$ is convergent to some limit $(z_n) \to l$. By Theorem 2.5.2
  we know that all subsequences of $(z_n)$ must converge to the same limit. Consider some $a \in x_n$ such that $x_i = a$. Note that $x_i = z_{2i-1}$
  and therefore $a \in z_n$. Similarly for some $b \in y_n$ such that $y_i = b$ we know that $y_i = z_{2i}$ and therefore $b \in z_n$. Thus both $x_n$ and $y_n$
  are subsequences of $z_n$ and therefore $\lim z_n = \lim x_n = \lim y_n$.
\end{proof}

\end{exercise}
\vspace{.5in}





\begin{exercise}{5} Suppose $F$ is a collection of open intervals such that if $I,J \in F$ and $I \neq J$, then $I \cap J = \emptyset$. Prove that $F$ is countable.\\

  \begin{proof}
    Suppose $F$ is a collection of open intervals such that each interval is disjoint. By Theorem 1.4.3 (The Density of $\Rats$ in $\Reals$) we 
    know that there must exist at least one rational number $r$ in side of each open interval. Define $A$ as a set containing the 
    compliment of the union of all the open sets in $F$,
    \begin{equation*}
     ( \bigcup_F J)^c \in A.
    \end{equation*}
    Note that $|A| = 1$, and that,
    \begin{equation*}
      (\bigcup_F J)\cup (\bigcup_A I) = \Reals.
    \end{equation*}
    Consider the function,
    $f: \Rats \to F \cup A$ such that $f(r) = J$ when $r \in J$. Now we will show that $f$ is a surjective function. Consider
    some $J \in F \cup A$, and note that by Theorem 1.4.3 there must exist some rational number $r \in J$ and thus $f$ is surjective. It then follows that $F \cup A$ is at most countable 
    and since $F$ and $A$ are also disjoint, $|F| = |F \cup A| - 1$. Thus $F$ is also at most countable. 
  \end{proof}

\end{exercise}

\vspace{.5in}




\begin{exercise}{6} Let $(x_n)$ be a sequence converging to $L$. Define,
  \begin{equation*}
    y_n = \dfrac{x_1+\dots+x_n}{n}.
  \end{equation*}
  That is $y_n$ is the average of the first $n$ terms of the $x_n$ sequence. Show that $y_n = L$ as well. \\

\begin{proof}
  Suppose that the sequence $(x_n)$ is convergent to $L$. Therefore by definition for all $\epsilon > 0$,
  \begin{equation*}
    |x_n - L|< \epsilon.
  \end{equation*} 
  Now consider the expression,
  \begin{equation*}
    |y_n - L| = |\dfrac{x_1+\dots+x_n}{n} - L|.
  \end{equation*}
  Through some algebra we get,
  \begin{align*}
    |y_n - L|&= \dfrac{1}{n}|(x_1+\dots+x_n) - nL|,\\
            &= \dfrac{1}{n}|(x_1 - L)+\dots+(x_n - L)|.
  \end{align*}
  By triangle inequality,
  \begin{equation*}
    |y_n - L|  \le \dfrac{1}{n}|(x_1 - L)|+\dots+|(x_n - L)|.
  \end{equation*}
  Since $(x_n)$ is convergent to $L$ we know that for all $\epsilon > 0$ there exists an $N\in \Nats$ such that for all $n \geq N$,
  \begin{equation*}
    |x_n - L|< \epsilon.
  \end{equation*} 
  By substitution we know that,
\begin{align*}
  |y_n - L| &< \dfrac{n \epsilon}{n},\\
  &< \epsilon.
\end{align*}
Thus $y_n$ is convergent with $y_n \to L$.
\end{proof}

\end{exercise}

\vspace{.5in}



\begin{exercise}{7} Use the Bolzano Weierstrass Theorem to prove the Monotone Convergence Theorem 
  without assuming any other form of the Axiom of Completeness.\\
  
  \begin{proof} Consider $(a_n)$, a monotone and bounded sequence. Without loss of generality let's assume
    the $(a_n)$ is monotone increasing. By Bolzano Weierstrass we know that there exists a convergent subsequence of $(a_n)$,
    $(a_{n_k}) \to L$. Therefore for all $\epsilon > 0$ there exists an $K \in \Nats$ such that for all $k \geq K$,
    \begin{equation*}
      |(a_{n_k})| < \epsilon.
    \end{equation*} 
    \begin{align*}
      |(a_{n_k}) - L| &< \epsilon\\
      -\epsilon < &(a_{n_k}) - L < \epsilon\\
      L - \epsilon < &(a_{n_k})< \epsilon + L
    \end{align*}
    Let $N = n_K$, note that for all $n \geq N$ there exists a $k \geq K$, such that
    \begin{equation*}
      a_{n_k} \le a_n \le a_{n_{k+1}}.
    \end{equation*}
    Thus we get the following inequality
    \begin{equation*}
      L - \epsilon < a_{n_k} \le a_n \le a_{n_{k+1}}< \epsilon + L.
    \end{equation*}
    Therefore $a_n$ converges to $L$. 
  \end{proof}




  \begin{exercise}{8} Suppose $(x_n)$ is a sequence and that for all $n \geq 2$,
    \begin{equation*}
      |x_{n+1} - x_n| \le \dfrac{1}{2} |x_n - x_{n-1}|.
    \end{equation*}
    Show that the sequence $(x_n)$ converges. \\

    \begin{proof}
      Suppose $(x_n)$ is a sequence and that for all $n \geq 2$,
      \begin{equation*}
        |x_{n+1} - x_n| \le \dfrac{1}{2} |x_n - x_{n-1}|.
      \end{equation*}
      Note that by expansion and iterative substitution, we get the following expression,
      \begin{equation*}
        |x_n - x_{n_1}| = \dfrac{1}{2^{n-1}}|x_2 - x_1|.
      \end{equation*}
      Let $M = |x_2 - x_1|$ and recall that in Homework 4, supplemental exercise 2 we proved that,
      \begin{equation*}
        \lim_{n \to \infty} \dfrac{1}{2^n} = 0. 
      \end{equation*}
      Thus it follows by Algebraic Limit Theorem that,
      \begin{equation*}
        \lim |x_n - x_{n_1}| = \lim \dfrac{M}{2^n} = M0 = 0.
      \end{equation*}
      Now consider the following, where $m,n \in \Nats$ and $m \geq n$,
      \begin{equation}
        |x_m - x_n| = |x_m - x_{m-1} + x_{m-1} - \dots + x_{n+1} - x_n|.
      \end{equation}
      By Triangle Inequality we get that,
      \begin{equation*}
        |x_m - x_n| \le \sum_{i = 1}^{m-n} |x_{1+n} - x_{1+n - 1}|.  
      \end{equation*}
      Using the given property we can get each term in the sum as a factor of $|x_{n+1} - x_n|$,
      \begin{equation*}
        |x_m - x_n| \le \sum_{i = 1}^{m-n} \dfrac{1}{2^{i-1}}|x_{n+1} - x_n|.
      \end{equation*}
      Since the sum is a constant term let,
      \begin{equation*}
        C = \sum_{i = 1}^{m-n} \dfrac{1}{2^{i-1}}.
      \end{equation*}
      Let $\epsilon > 0$. Consider an $N\in \Nats$ such that for all $n\geq N$,
      \begin{equation*}
        |x_{n+1} - x_n| < \dfrac{\epsilon}{C}. 
      \end{equation*}
      Therefore we get the following, 
      \begin{align*}
        |x_m - x_n| &\le C |x_{n+1} - x_n|,\\
        &< C \dfrac{\epsilon}{C},\\
        &< \epsilon.
      \end{align*} 
      Therefore the sequence $x_n$ is Cauchy and converges.
    \end{proof}
  \end{exercise}

\end{exercise}


\begin{exercise}{9} Let $(a_n)$ and $(b_n)$ be sequences with $b_n \geq 0$
  for all $n$ and and $\lim_nb_n = 0$. We say that 
  $a_n = O(b_n)$ If there is a constsnt $C$ such thtat $|a_n|\le Cb_n$ for all $n$. Roughly
  speaking, $a_n = O(b_n)$ if the sequence $a_n$ converges to zero at least as fast as teh sequence $b_n$.\\
  Suppose $a_n$ and $b_n$ are sequences with $b_n > 0$. Suppose also that $\lim frac{a_n}{b_n} = L$ for some number $L$.
  Prove that $a_n = O(b_n)$. 







\end{exercise}








\end{document}