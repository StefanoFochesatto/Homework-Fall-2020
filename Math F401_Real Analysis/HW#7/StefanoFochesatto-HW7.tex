%%%%%%%%%%%%%%%%%%%%%%%%%%%%%%%%%%%%%%%%%%%%%%%%%%%%%%%%%%%%%%%%%%%%%%%%%%%%%%%%%%%%%%%
%%%%%%%%%%%%%%%%%%%%%%%%%%%%%%%%%%%%%%%%%%%%%%%%%%%%%%%%%%%%%%%%%%%%%%%%%%%%%%%%%%%%%%%
% 
% This top part of the document is called the 'preamble'.  Modify it with caution!
%
% The real document starts below where it says 'The main document starts here'.

\documentclass[12pt]{article}

\usepackage{amssymb,amsmath,amsthm}
\usepackage[top=1in, bottom=1in, left=1.25in, right=1.25in]{geometry}
\usepackage{fancyhdr}
\usepackage{enumerate}

% Comment the following line to use TeX's default font of Computer Modern.
\usepackage{times,txfonts}

\newtheoremstyle{homework}% name of the style to be used
  {18pt}% measure of space to leave above the theorem. E.g.: 3pt
  {12pt}% measure of space to leave below the theorem. E.g.: 3pt
  {}% name of font to use in the body of the theorem
  {}% measure of space to indent
  {\bfseries}% name of head font
  {:}% punctuation between head and body
  {2ex}% space after theorem head; " " = normal interword space
  {}% Manually specify head
\theoremstyle{homework} 

% Set up an Exercise environment and a Solution label.
\newtheorem*{exercisecore}{Exercise \@currentlabel}
\newenvironment{exercise}[1]
{\def\@currentlabel{#1}\exercisecore}
{\endexercisecore}

\newcommand{\localhead}[1]{\par\smallskip\noindent\textbf{#1}\nobreak\\}%
\newcommand\solution{\localhead{Solution:}}

%%%%%%%%%%%%%%%%%%%%%%%%%%%%%%%%%%%%%%%%%%%%%%%%%%%%%%%%%%%%%%%%%%%%%%%%
%
% Stuff for getting the name/document date/title across the header
\makeatletter
\RequirePackage{fancyhdr}
\pagestyle{fancy}
\fancyfoot[C]{\ifnum \value{page} > 1\relax\thepage\fi}
\fancyhead[L]{\ifx\@doclabel\@empty\else\@doclabel\fi}
\fancyhead[C]{\ifx\@docdate\@empty\else\@docdate\fi}
\fancyhead[R]{\ifx\@docauthor\@empty\else\@docauthor\fi}
\headheight 15pt

\def\doclabel#1{\gdef\@doclabel{#1}}
\doclabel{Use {\tt\textbackslash doclabel\{MY LABEL\}}.}
\def\docdate#1{\gdef\@docdate{#1}}
\docdate{Use {\tt\textbackslash docdate\{MY DATE\}}.}
\def\docauthor#1{\gdef\@docauthor{#1}}
\docauthor{Use {\tt\textbackslash docauthor\{MY NAME\}}.}
\makeatother

% Shortcuts for blackboard bold number sets (reals, integers, etc.)
\newcommand{\Reals}{\ensuremath{\mathbb R}}
\newcommand{\IRats}{\ensuremath{\mathbb I}}
\newcommand{\Nats}{\ensuremath{\mathbb N}}
\newcommand{\Ints}{\ensuremath{\mathbb Z}}
\newcommand{\Rats}{\ensuremath{\mathbb Q}}
\newcommand{\Cplx}{\ensuremath{\mathbb C}}
%% Some equivalents that some people may prefer.
\let\RR\Reals
\let\NN\Nats
\let\II\Ints
\let\CC\Cplx

%%%%%%%%%%%%%%%%%%%%%%%%%%%%%%%%%%%%%%%%%%%%%%%%%%%%%%%%%%%%%%%%%%%%%%%%%%%%%%%%%%%%%%%
%%%%%%%%%%%%%%%%%%%%%%%%%%%%%%%%%%%%%%%%%%%%%%%%%%%%%%%%%%%%%%%%%%%%%%%%%%%%%%%%%%%%%%%
% 
% The main document start here.

% The following commands set up the material that appears in the header.
\doclabel{Math 401: Homework 7}
\docauthor{Stefano Fochesatto}
\docdate{\today}

\begin{document}

Light homework.  You've been working hard!

\begin{exercise}{Supplemental 1} Write up a nice proof of the Alternating
Series Test.

\begin{proof} Suppose $a_n$ is monotone decreasing and converges, $a_n \to 0$. We will demonstrate
  that the sum,
  \begin{equation*}
    \sum_{n = 1}^{\infty} (-1)^{n+1}a_n
  \end{equation*}
  Converges by showing that the partial sums with an even number of terms, $s_{2j}$ and the partial
  sums with an odd number of terms, $s_{2j+1}$ converge using the Monotone Convergence Theorem. First consider the
  sequence of partial sums with an odd amount of terms, $s_{2j+1}$. Consider the $j+1^{th}$ term,
  \begin{equation*}
    s_{2{j+1}+1} = s_{2{j}+1} - a_{2j + 2} + a_{2{j+1}+1}.
  \end{equation*}
Since we know that the sequence $a_n$ is monotone decreasing we know that $a_{2j + 2} \geq a_{2{j+1}+1}$ and thus it follows that,
$s_{2j+1}$ is monotone decreasing. By construction we also see that $a_1 - a_2$ is a lower bound for the series $s_{2j+1}$ thus by MCT it converges to $L'$. \\

Similarly consider the sequence of partial sums with an even number of terms ,$s_{2j}$. First note that the $s_{2(j+1}$ term,
\begin{equation*}
  s_{2(j+1)} = s_{2j} + a_{2j+1} - a_{2(j+1)}.
\end{equation*} 
Again since the sequence $a_n$ is monotone decreasing we know that $a_{2j+1} \geq a_{2(j+1)}$ and thus it follows that,
$s_{2j}$ is monotone increasing. By construction we also see that $a_1$ is an upper bound for the series $s_{2j}$ thus by MCT it converges to $L$.\\

Now we will show that both even and odd subsequences converge to the same limit. Consider the following true expression,
\begin{equation*}
  s_{2j} + a_{2j+1} = s_{2j+1} 
\end{equation*}
Now consider taking the limit of both sequences,
\begin{equation*}
  \lim (s_{2j} + a_{2j+1}) = \lim(s_{2j+1}) 
\end{equation*}
By the ALT and the fact that $a_n \to 0$ we know that,
\begin{equation*}
  \lim (s_{2j}) + \lim(a_{2j+1}) = \lim(s_{2j+1})
\end{equation*}
\begin{equation*}
  L + 0 = L'
\end{equation*}
Therefore since both the even and odd sub sequences converge to the same limit the sequence of all partial sums must converge.

%% Show that both the limits are the same
\end{proof}

\end{exercise}

\begin{exercise}{Exercise 2.7.9} Given the series $\sum_{n = 1}^{\infty} a_n$
  with $a_n \neq 0$, the Ratio Test states that if $a_n$ satisfies,
  \begin{equation*}
    \lim{\dfrac{a_{n+1}}{a_n}} = r < 1,
  \end{equation*}
  Then the series converges absolutely.\\


\begin{enumerate}
  \item Let $r'$ satisfy $r < r' < 1$. Explain why there exists an $N$ such that 
  $n \le N$ implies $|a_{n+1}|\le |a_n|r'$.\\

  \solution By the Ratio Test and the definition of convergence we know that for all 
  $\epsilon > 0$ there exists an $N \in \Nats$ such that for all $n \geq N$,
  \begin{equation*}
    ||\dfrac{a_{n+1}}{a_n}| - r| < \epsilon 
  \end{equation*}
  Note that since $r' > r$ we know that $r' - r > 0$. Now we let $\epsilon  = r' - r$, and then through some algebra we get,
  \begin{align*}
    ||\dfrac{a_{n+1}}{a_n}| - r| &< r' - r,\\
    |\dfrac{a_{n+1}}{a_n}| - r &< r' - r,\\
    |\dfrac{a_{n+1}}{a_n}| &< r', \\
    \dfrac{|a_{n+1}|}{|a_n|} &< r', \\
    |a_{n+1}| &< r'|a|.
  \end{align*}
  \vspace{.25in}


  \item Why does $|a_N| \sum (r')^n$ converge\\
  
  \solution Note that by Example 2.7.5 we know that $\sum |a_N| (r')^n$ is a convergent geometric series, since $r < r' <1$.

  \vspace{.25in}


  \item Now show that if $\sum|a_n|$ converges and conclude that $\sum a_n$ converges.\\
  
  \begin{proof}
    By the supposition of the ratio test we found in part 1 that for some $N \in Nats$ then for all $r < r' < 1$ and $n \geq N$,
    \begin{equation*}
      |a_{n+1}| \le |a_n|r'.
    \end{equation*}
    Therefore we can surmise that for all $i \geq 0$ and $n \geq N$,
    \begin{equation*}
      |a_{n+i}| \le |a_n|r'^i.
    \end{equation*} 
    Now consider the following sum,
    \begin{equation*}
      \sum_{i = 1}^{\infty} |a_i| = \sum_{i = 1}^{\infty} |a_{N+1}| \le \sum_{i = 1}^{\infty} |a_N|r'^i 
    \end{equation*}
    Recall that in part 2 we showed that the right had side of the inequality converges, therefore by the Comparison Test
    we know that, $\sum|a_n|$ and by the Absolute convergence test we know that $\sum a_n$ must also converge.
  \end{proof}

\end{enumerate}

\end{exercise}


\vspace{.5in}

\begin{exercise}{3.2.2} Let,
  \begin{equation*}
    A = \{(-1)^n + \dfrac{2}{n}| n \in Nats \},
  \end{equation*}
  \begin{equation*}
    B = \{x \in \Rats | 0 < x < 1 \}.
  \end{equation*}
  \begin{enumerate}
    \item What are the limit points.\\
    
    \solution Consider the following two subsequences of $A$,
    \begin{equation*}
      A_{odd} = (-1)^{2n+1} + \dfrac{2}{2n+1}
    \end{equation*}
    \begin{equation*}
      A_{even} = (-1)^{2n} + \dfrac{2}{2n}
    \end{equation*}
    Note that $A_{odd} \to -1$ and $A_{even} \to 1$ therefore by Theorem 3.2.5, 1 and -1 are limit points.\\

    For $B$ consider the Density of $\Rats$ in $\Reals$ and we see that for all $x \in (0,1)$ there exists a sequence $a_n \in \Rats$ and therefore by 
    Theorem 3.2.5 we know that all $x \in (0,1)$ are limit points.

    \vspace{.25in}

    \item Is the set open? or closed?\\
    
    For set $A$ consider the $\sup A = 2$ and note that for all $V_{\epsilon}(2)$ contain upper bounds of $A$ thus $V_{\epsilon}(2) \not\subseteq A$.\\

    For set $B$ consider the Density of $\Reals - \Rats$ in $\Reals$, and note that for all $x \in B$ there must exist an irrational number $i \in V_{\epsilon}(2)$.
    \vspace{.25in}

    \item Does the set contain any isolated points?\\
    \solution $A$ is composed of all isolated points except for $\{1\}$ and $B$ is composed of all it's rational limit points therefore non are isolated. 
    \vspace{.25in}

    \item Find the closure of the set\\
    \solution $\overline{A} = A \cup \{-1\}$\\
    $\overline{B} = 0 < x < 1$
  \end{enumerate}
\end{exercise}


\begin{exercise}{3.2.4} Let $A$ be nonempty and bounded above so that $s = \sup A$ exists. \\
  \begin{enumerate}
    \item Show that $s \in \overline{A}$.\\
    
    \begin{proof}
      Suppose $A$ is nonempty and bounded above so that $s = \sup A$ exists. Consider that sequence \
      \begin{equation*}
        a_n = s - \dfrac{1}{n}.
      \end{equation*}
      Clearly the sequence $a_n$ converges to $s$, (a quick proof by def will show this) and since $s$ is a supremum
      there must some sub sequence of $a_n$ that is contained in $A$ which also converges to $s$. Thus by Theorem 3.2.5
       we know that $s$ is a limit point and by definition $s \in \overline{A}$.\\
    \end{proof}

    
    \vspace{.25in}



    \item Can an open set contain its supremum.\\
    
    \begin{proof}
      Suppose for the sake of contradiction that there exists an open set $A$ where $\sup A = s$ and $s \in A$.
      By the definition of open we know that for all $a \in A$ there exists an $\epsilon-neighborhood$ such that,
      \begin{equation*}
        V_{\epsilon}(a) \subseteq A.
      \end{equation*}
      Now consider the $V_{\epsilon}(s)$ and recall that any $x \in \Reals$ where $x>s$ we know that $x \notin A$ and therefore it must be the case that
      \begin{equation*}
        V_{\epsilon}(s) \not\subseteq A.
      \end{equation*} 
      Thus an open set cannot contain its supremum.  

    \end{proof}
    





  \end{enumerate} 
\end{exercise}

\end{document}