%%%%%%%%%%%%%%%%%%%%%%%%%%%%%%%%%%%%%%%%%%%%%%%%%%%%%%%%%%%%%%%%%%%%%%%%%%%%%%%%%%%%%%%
%%%%%%%%%%%%%%%%%%%%%%%%%%%%%%%%%%%%%%%%%%%%%%%%%%%%%%%%%%%%%%%%%%%%%%%%%%%%%%%%%%%%%%%
% 
% This top part of the document is called the 'preamble'.  Modify it with caution!
%
% The real document starts below where it says 'The main document starts here'.

\documentclass[12pt]{article}

\usepackage{amssymb,amsmath,amsthm}
\usepackage[top=1in, bottom=1in, left=1.25in, right=1.25in]{geometry}
\usepackage{fancyhdr}
\usepackage{enumerate}

% Comment the following line to use TeX's default font of Computer Modern.
\usepackage{times,txfonts}

\newtheoremstyle{homework}% name of the style to be used
  {18pt}% measure of space to leave above the theorem. E.g.: 3pt
  {12pt}% measure of space to leave below the theorem. E.g.: 3pt
  {}% name of font to use in the body of the theorem
  {}% measure of space to indent
  {\bfseries}% name of head font
  {:}% punctuation between head and body
  {2ex}% space after theorem head; " " = normal interword space
  {}% Manually specify head
\theoremstyle{homework} 

% Set up an Exercise environment and a Solution label.
\newtheorem*{exercisecore}{Exercise \@currentlabel}
\newenvironment{exercise}[1]
{\def\@currentlabel{#1}\exercisecore}
{\endexercisecore}

\newcommand{\localhead}[1]{\par\smallskip\noindent\textbf{#1}\nobreak\\}%
\newcommand\solution{\localhead{Solution:}}

%%%%%%%%%%%%%%%%%%%%%%%%%%%%%%%%%%%%%%%%%%%%%%%%%%%%%%%%%%%%%%%%%%%%%%%%
%
% Stuff for getting the name/document date/title across the header
\makeatletter
\RequirePackage{fancyhdr}
\pagestyle{fancy}
\fancyfoot[C]{\ifnum \value{page} > 1\relax\thepage\fi}
\fancyhead[L]{\ifx\@doclabel\@empty\else\@doclabel\fi}
\fancyhead[C]{\ifx\@docdate\@empty\else\@docdate\fi}
\fancyhead[R]{\ifx\@docauthor\@empty\else\@docauthor\fi}
\headheight 15pt

\def\doclabel#1{\gdef\@doclabel{#1}}
\doclabel{Use {\tt\textbackslash doclabel\{MY LABEL\}}.}
\def\docdate#1{\gdef\@docdate{#1}}
\docdate{Use {\tt\textbackslash docdate\{MY DATE\}}.}
\def\docauthor#1{\gdef\@docauthor{#1}}
\docauthor{Use {\tt\textbackslash docauthor\{MY NAME\}}.}
\makeatother

% Shortcuts for blackboard bold number sets (reals, integers, etc.)
\newcommand{\Reals}{\ensuremath{\mathbb R}}
\newcommand{\Nats}{\ensuremath{\mathbb N}}
\newcommand{\Ints}{\ensuremath{\mathbb Z}}
\newcommand{\Rats}{\ensuremath{\mathbb Q}}
\newcommand{\Cplx}{\ensuremath{\mathbb C}}
\newcommand{\Irats}{\ensuremath{\mathbb I}}
%% Some equivalents that some people may prefer.
\let\RR\Reals
\let\NN\Nats
\let\II\Ints
\let\CC\Cplx

%%%%%%%%%%%%%%%%%%%%%%%%%%%%%%%%%%%%%%%%%%%%%%%%%%%%%%%%%%%%%%%%%%%%%%%%%%%%%%%%%%%%%%%
%%%%%%%%%%%%%%%%%%%%%%%%%%%%%%%%%%%%%%%%%%%%%%%%%%%%%%%%%%%%%%%%%%%%%%%%%%%%%%%%%%%%%%%
% 
% The main document start here.

% The following commands set up the material that appears in the header.
\doclabel{Math 401: Homework 9}
\docauthor{Stefano Fochesatto}
\docdate{November 2, 2020}

\begin{document}

\begin{exercise}{Abbott 4.3.9} Assume $h : \Reals \to \Reals$ is continuous on $\Reals$ and let $k = \{x : h(x) = 0\}$. Show that $k$
  is a closed set.\\
  \begin{proof} Suppose $h : \Reals \to \Reals$ is continuous on $\Reals$ and $k = \{x : h(x) = 0\}$. Let $x$ be a limit point of $k$, by 
    Theorem 3.2.5 there exists a sequence $(a_n) \in k$ such that $lim a_n = x$ where $a_n \neq x$ for all $n \in \Nats$. By Theorem 4.3.2 (iii) since $h$ is continuous
    for all $(a_n) \to x$ it follows that $h(a_n) \to h(x)$. Note that since $a_n \in k$ we know that $h(a_n) = 0$ for all $n$ and therefore we know that $h(x) = 0$ an thus by definition we get that $x \in k$. 
    Thus $k$ contains all its limit points and is therefore closed. 
  \end{proof}
\end{exercise}

\begin{exercise}{Supplemental 1}
\begin{enumerate}[a)]
\item Show that a continuous function on all of $\Reals$ that equals zero
on the rational numbers must be the zero function\\
\begin{proof}
Suppose $f : \Reals \to \Reals$ is continuos and that for all $q \in \Rats$ we know that $f(q) = 0$. By Theorem 3.2.10 for every $x \in \Reals$
there exists a sequence $(q_n) \in \Rats$ such that $(q_n) \to x$. By the continuity of $f$ we know that $f(q_n) \to f(x)$, since all $q_n \in \Rats$ by definition of $f$
we know that $f(q_n) = 0$ and thus $f(x) = 0$ for all $x \in \Reals$.
\end{proof}
\vspace{.25in}

\item Suppose $f$ and $g$ are two continuous functions on the real
numbers. Is it true that if $f(q)=g(q)$ for all $q\in\Rats$, then $f$ and $g$
are the same function?\\
\begin{proof}
  Suppose $f$ and $g$ are two continuous functions on the real numbers such that $f(q)=g(q)$ for all $q\in\Rats$. By Theorem 3.2.10 fir ever $x \in \Reals$
  there exists a $(q_n) \in \Rats$ such that $(q_n) \to x$. By the continuity of $f$ and $g$ we know that $f(q_n) \to f(x)$ and  $g(q_n) \to g(x)$. Since $f(q) = g(q)$
  we also get that $g(q_n) \to f(x)$ and $f(q_n) \to g(x)$. Finally by Theorem 2.2.7(Uniqueness of Limits) it must be the case that the limits are the same and we get 
  $f(x) = g(x)$ for all $x \in \Reals$.
\end{proof}
\end{enumerate}	
\end{exercise}
\vspace{.5in}




\begin{exercise}{Supplemental 2} 
 Suppose $K\subseteq \Reals$ is compact.  Show that
there exists $x_M\in K$ such that $x_M\ge x$ for all $x\in K$.  Then,
with very little work, show that there exists $x_m\in K$ such
that $x_m \le x$ for all $x\in K$.

\begin{proof}
  Suppose that $K\subseteq \Reals$ is compact. By the definition of compact we know that $k$ is closed and bounded.
  Since $K$ is bounded we know that there exists some $x_M = Sup K$. Now consider every $\epsilon$-neighborhood of $x_M$. By 
  Lemma 1.3.8 we know that for every $\epsilon>0$, there exists some $x \in K$ such that $x_M - \epsilon < x < x_M$. So we know that $x \in V_\epsilon(x_M)\cap K/\{x_M\}$ and thus by definition $x_M$ is a limit point of $K$
  and since $K$ is closed we know that $x_M \in K$\\

  Since $K$ is bounded there also exists an $x_m = Inf K$. By Lemma 1.3.8 we know that for every $\epsilon>0$, there exists some $x \in K$ such that $x_m < x < x_m + \epsilon$
  So we know that $x \in V_\epsilon(x_m)\cap K/\{x_m\}$ and thus by definition $x_m$ is a limit point of $K$ and since $K$ is closed we know that $x_m \in K$
\end{proof}
\end{exercise}
\vspace{.5in}






\begin{exercise}{Abbott 4.3.7(a)}Referring to the proper theorems, 
give a formal argument that Dirichlet’s function from Section 4.1 is nowhere-continuous on $\Reals$.\\

\begin{proof}
  Consider the Dirichlet’s function $f: \Reals \to \Reals$ where,
  \begin{equation*}
    f(x) = 
    \begin{cases} 
      1 & x \in \Rats \\
      0 & x \notin \Rats 
   \end{cases}
  \end{equation*}
  Consider the $i \in \Irats$, by Theorem 3.2.10 (Density of $\Rats$ in $\Reals$) we can construct a sequence
  $q_n \in \Rats$ such that $q_n \to i$. Suppose for the sake of contradiction that $f$ is continuous on all $i \in \Irats$ then
  by continuity it must be the case that since $q_n \to i$ then $f(q_n) \to f(i)$, however $f(q_n)$ is a constant sequence of 1 and $f(i) = 0$ therefore $f(q_n) \not\to f(i)$. Thus by contradiction $f$ is not continuous on $\Irats$

Similarly consider $q \in \Rats$, and constructing a sequence $i_n \in \Irats$ such that $i_n \to q$. Supposing that $f$ is continuous on all $q \in \Rats$ then by continuity
we it must be the case that since $i_n \to q$ then $f(i_n) \to f(q)$ however $f(i_n)$ is a constant sequence of 0 and $f(q) = 1$ therefore by contradiction $f$ is not continuous on $\Rats$.
\end{proof}
\end{exercise}
\vspace{.5in}





\begin{exercise}{Abbott 4.4.6} Give an example of each of the following, or state that such a request is impossible.
  for any that are impossible, supply a short explanation for why this is the case.\\

  \begin{enumerate}
    \item A continuous function $f: (0,1) \to \Reals$ and a Cauchy sequence $(x_n)$ such that $f(x_n)$ is not a cauchy sequence.\\
    \solution Let $f(x) = \frac{1}{x}$ and consider the cauchy sequence $x_n =\frac{1}{n!}$. Note that $f(x_n) = n!$ which is clearly not convergent and is therefore not a cauchy sequence. 
    \vspace{.25in}

    \item A uniformly continuous function $f: (0,1) \to \Reals$ and a Cauchy sequence $(x_n)$ such that $f(x_n)$ is not a cauchy sequence.\\
    \solution Suppose $f: (0,1) \to \Reals$ and a Cauchy sequence $(x_n)$ and for the sake of contradiction suppose $f(x_n)$ is not cauchy. By definition there exists some
    (bad) $epsilon$ such that for every $N \in \Nats$ for all $n,m \geq N$ we know that $|f(x_n) - f(x_m)| \geq \epsilon$. However since $x_n$ is cauchy and $f$ is continuous we know that 
    $|x_n - x_m| \to 0$ then $|f(x_n) - f(x_m)| \to 0$ thus by contradiction $f$ cannot be uniformly continuous.
    \vspace{.25in}

    \item A continuous function $f: [0,\infty) \to \Reals$ and a Cauchy sequence $(x_n)$ such that $f(x_n)$ is not a cauchy sequence.\\
    \solution This request is impossible, note that the set $[0,\infty)$ is closed and therefore by Theorem 3.2.8 we know that $x_n \to L$ where $L \in [0,\infty)$.
    By continuity we get that $f(x_n) \to f(L)$ and since $f(x_n)$ is a convergent sequence it is also cauchy. 
  \end{enumerate}
\end{exercise} 
\vspace{.5in}


\begin{exercise}{Abbott 4.4.9} A function $f: A \to \Reals$ is called $Lipschitz$ is there exists a bound $M>0$ such that,
  \begin{equation*}
    |\dfrac{f(x) - f(y)}{x - y}|\le M.
  \end{equation*}
for all $x \neq y \in A$. Geometrically speaking, a function $f$ is $Lipschitz$ if there is a uniform bound on the magnitude
of the sloped of lines drawn through any two points on the graph of $f$,\\


\begin{enumerate}
  \item Show that if $f: A \to \Reals$ is $Lipschitz$, then it is uniformly continuous on $A$.\\
  \begin{proof}
    Suppose that $f: A \to \Reals$ is $Lipschitz$. By the definition of a $Lipschitz$ function we know that for all $x,y \in A$ there
    exists some $M$ such that,
    \begin{equation*}
      |f(x) - f(y)| \le M|x - y|.
    \end{equation*}
    Let $\epsilon > 0$, now consider $\delta = \frac{\epsilon}{M}$ therefore whenever $|x - y|<\delta$ we get,
    \begin{align*}
      |x - y| &< \delta,\\
      M|x - y| &< M\delta,\\
      |f(x) - f(y)| &< \epsilon.
    \end{align*}
    Thus $f$ is uniformly continuous on $A$.
  \end{proof}
  \vspace{.25in}



  \item Is the converse statement true? Are all uniformly continuous functions necessarily $Lipschitz$?\\
  \begin{proof}
    Consider the function $f: [0,1] \to [0,1]$ such that $f(x) =\sqrt{x}$. We demonstrated in class that $f$ is uniformly continuous. Suppose 
    for the sake of contradiction that $f$ is $Lipschitz$, we would get the following inequality for all $x,y \in [0,1]$ and some $M>0$,
    \begin{equation*}
      |\sqrt{x} - \sqrt{y}|\le M|x - y|
    \end{equation*}
    Let $x = 0$ and $y = \frac{1}{2M^2}$ we get the following by substitution,
    \begin{align*}
      |\sqrt{x} - \sqrt{y}|&\le M|x - y|,\\
      \frac{1}{2M}&\le \frac{M}{4M^2},\\
      \frac{1}{2M}&\le \frac{1}{4M}.
    \end{align*}
    Thus by contradiction $f$ is not $Lipschitz$. As demonstrated in class we can see that as $x$ approaches 0 the slope increases to infinity and is therefore unbounded.  
  \end{proof}
  








\end{enumerate}





\end{exercise}

\end{document}