%%%%%%%%%%%%%%%%%%%%%%%%%%%%%%%%%%%%%%%%%%%%%%%%%%%%%%%%%%%%%%%%%%%%%%%%%%%%%%%%%%%%%%%
%%%%%%%%%%%%%%%%%%%%%%%%%%%%%%%%%%%%%%%%%%%%%%%%%%%%%%%%%%%%%%%%%%%%%%%%%%%%%%%%%%%%%%%
% 
% This top part of the document is called the 'preamble'.  Modify it with caution!
%
% The real document starts below where it says 'The main document starts here'.

\documentclass[12pt]{article}

\usepackage{amssymb,amsmath,amsthm}
\usepackage[top=1in, bottom=1in, left=1.25in, right=1.25in]{geometry}
\usepackage{fancyhdr}
\usepackage{enumerate}

% Comment the following line to use TeX's default font of Computer Modern.
\usepackage{times,txfonts}

\newtheoremstyle{homework}% name of the style to be used
  {18pt}% measure of space to leave above the theorem. E.g.: 3pt
  {12pt}% measure of space to leave below the theorem. E.g.: 3pt
  {}% name of font to use in the body of the theorem
  {}% measure of space to indent
  {\bfseries}% name of head font
  {:}% punctuation between head and body
  {2ex}% space after theorem head; " " = normal interword space
  {}% Manually specify head
\theoremstyle{homework} 

% Set up an Exercise environment and a Solution label.
\newtheorem*{exercisecore}{Exercise \@currentlabel}
\newenvironment{exercise}[1]
{\def\@currentlabel{#1}\exercisecore}
{\endexercisecore}

\newcommand{\localhead}[1]{\par\smallskip\noindent\textbf{#1}\nobreak\\}%
\newcommand\solution{\localhead{Solution:}}

%%%%%%%%%%%%%%%%%%%%%%%%%%%%%%%%%%%%%%%%%%%%%%%%%%%%%%%%%%%%%%%%%%%%%%%%
%
% Stuff for getting the name/document date/title across the header
\makeatletter
\RequirePackage{fancyhdr}
\pagestyle{fancy}
\fancyfoot[C]{\ifnum \value{page} > 1\relax\thepage\fi}
\fancyhead[L]{\ifx\@doclabel\@empty\else\@doclabel\fi}
\fancyhead[C]{\ifx\@docdate\@empty\else\@docdate\fi}
\fancyhead[R]{\ifx\@docauthor\@empty\else\@docauthor\fi}
\headheight 15pt

\def\doclabel#1{\gdef\@doclabel{#1}}
\doclabel{Use {\tt\textbackslash doclabel\{MY LABEL\}}.}
\def\docdate#1{\gdef\@docdate{#1}}
\docdate{Use {\tt\textbackslash docdate\{MY DATE\}}.}
\def\docauthor#1{\gdef\@docauthor{#1}}
\docauthor{Use {\tt\textbackslash docauthor\{MY NAME\}}.}
\makeatother

% Shortcuts for blackboard bold number sets (reals, integers, etc.)
\newcommand{\Reals}{\ensuremath{\mathbb R}}
\newcommand{\Nats}{\ensuremath{\mathbb N}}
\newcommand{\Ints}{\ensuremath{\mathbb Z}}
\newcommand{\Rats}{\ensuremath{\mathbb Q}}
\newcommand{\Cplx}{\ensuremath{\mathbb C}}
%% Some equivalents that some people may prefer.
\let\RR\Reals
\let\NN\Nats
\let\II\Ints
\let\CC\Cplx

%%%%%%%%%%%%%%%%%%%%%%%%%%%%%%%%%%%%%%%%%%%%%%%%%%%%%%%%%%%%%%%%%%%%%%%%%%%%%%%%%%%%%%%
%%%%%%%%%%%%%%%%%%%%%%%%%%%%%%%%%%%%%%%%%%%%%%%%%%%%%%%%%%%%%%%%%%%%%%%%%%%%%%%%%%%%%%%
% 
% The main document start here.

% The following commands set up the material that appears in the header.
\doclabel{Math 401: Take Home Final}
\docauthor{Stefano Fochesatto}
\docdate{\today}

\begin{document}

%% Some case of the ratio test or l'hospitals rule
\begin{exercise}{1} Suppose the $(x_n)$ and $(y_n)$ are sequences that $\lim_{n \to  \infty} x_n = L$ and $\lim_{n \to  \infty} y_n = \infty$.
  Show that $\lim_{n \to  \infty} x_n/y_n = 0$. 
\begin{proof}
  Suppose that $(x_n)$ and $(y_n)$ are sequences that $\lim_{n \to  \infty} x_n = L$ and $\lim_{n \to  \infty} x_n = \infty$.
  First we will show that the series $\frac{1}{y_n}$ converges to 0. Let $\epsilon > 0$.  Let $\epsilon > \frac{1}{y_N}$. Since $y_n$ diverges there must exists some 
  $N \geq n$ such that $\frac{1}{y_n} \geq \frac{1}{y_N}$, therefore,
  \begin{align*}
    |\frac{1}{y_n} - 0| &= \frac{1}{y_n},\\
    &\le \frac{1}{y_N},\\
    &< \epsilon. 
  \end{align*}
  Thus $\lim_{n \to  \infty} \frac{1}{y_n} = 0$. By ALT we know that $\lim_{n \to  \infty} x_n/y_n = L(0) = 0$.
\end{proof}


\end{exercise}
\vspace{.5in}




%% Induction, true on order 0 and 1. Suppose its true for order n. Consider an order n+1 polynomial f = (x - a)q(x). By The induction hypothesis,
%% we know that f has at most n roots.
\begin{exercise}{2} A number is algebraic if it is a solution of a polynomial equation,
  \begin{equation*}
    a_nx^n + a_{n-1}x^{n-1}+\dots+a_1x + a_0 = 0
  \end{equation*}
  Where each $a_k$ is an integer, $n \geq 1$ and $a_n \neq 0$. Show that the collection of all algebraic 
  number is countable.\\

  \begin{proof}
    Suppose an $n$ degree polynomial with a non-zero leading term. Note that the set of roots to this polynomial
    is finite and by the Fundamental Theorem of Algebra there are at most $n$ real roots. Let $A_n$ be the set of all
    solutions to an $n$ degree polynomial with a non-zero leading term. Note that since $a_n \in \Ints$, by the multiplication property an upper 
    bound on $|A_n|$ is given by,
    \begin{equation*}
      |A_n| \le n|\Ints|^n.
    \end{equation*} 
    Since $A_n$ is bounded above by a countable product, $A_n$ must also be countable. By Definition the set of all algebraic
    numbers is the set of all roots of all polynomial equations that have order $n \geq 1$ and non- zero leading terms. Thus 
    the set of algebraic numbers is given by,
    \begin{equation*}
      \bigcup_{i = 1}^{\infty} A_i.
    \end{equation*}
    Therefore by Theorem 1.5.8 (ii) the set of all algebraic numbers is countable. 





  \end{proof}


\end{exercise}
\vspace{.5in}


\begin{exercise}{3} Let $p$ be a fifth order polynomial, so $p(x) = \sum_{k = 0}^5 a_kx^k$ where each 
  $a_k \in \Reals$, and $a_5 \neq 0$. Prove that there exists a solution of $p(x) = 0$.\\

  \begin{proof}
    Suppose that $p$ be a fifth order polynomial, so $p(x) = \sum_{k = 1}^5 a_kx^k$ where each 
    $a_k \in \Reals$, and $a_5 \neq 0$. Consider the following factorization of $p$,
    \begin{equation*}
      p(x) = x^{5}(a_5 + \frac{a_4}{x} + \frac{a_3}{x^2} + \frac{a_2}{x^3} + \frac{a_1}{x^4} + \frac{a_0}{x^5}).
    \end{equation*}
    By the ALT we can see that the limit as $x \to \pm \infty$ of the summand term, we get,
    \begin{equation*}
      \lim_{x \to \infty} (a_5 + \frac{a_4}{x} + \frac{a_3}{x^2} + \frac{a_2}{x^3} + \frac{a_1}{x^4} + \frac{a_0}{x^5}) = a_5,
    \end{equation*}
    \begin{equation*}
      \lim_{x \to -\infty} (a_5 + \frac{a_4}{x} + \frac{a_3}{x^2} + \frac{a_2}{x^3} + \frac{a_1}{x^4} + \frac{a_0}{x^5}) = a_5.
    \end{equation*}
    Looking at the limit of $p(x)$ as $x \to \pm \infty$ we see that,
    \begin{equation*}
      \lim_{x \to \infty} p(x) = x^{5} a_5 = \infty,
    \end{equation*}
    \begin{equation*}
      \lim_{x \to -\infty} p(x) = x^{5} a_5 = -\infty. 
    \end{equation*}
    Thus there exists some $a, b \in \Reals$ such that $f(a)<0$ and $f(b)>0$. Note that since $p(x)$ is a polynomial 
    it is continuous on the domain $[a,b]$. Since $f(a)<0<f(b)$, then by the Intermediate Value Theorem there must exists a $c \in (a,b)$
    such that $p(c) = 0$. 
  \end{proof}

\end{exercise}
\vspace{.5in}




%Algebraic Limit theorem then use a p seris argument. Look for a proof of the root test
\begin{exercise}{4} Let $\sum_{k = 1}^{\infty}a_k$ be a series. Suppose moreover that $\lim_{k \to \infty} |a_k|^{\frac{1}{k}}$
  exists and equals $L$. Show that the series converges absolutely if $L<1$ and diverges if $L>1$. \\
\begin{proof}
  Suppose the series $\sum_{k = 1}^{\infty}a_k$ and that the $\lim_{k \to \infty} |a_k|^{\frac{1}{k}}$
  exists and equals $L$. Note that by the ALT we know that since $\lim_{k \to \infty} |a_k|^{\frac{1}{k}} = L$
  \begin{equation*}
    \lim_{k \to \infty} {|a_k|^{\frac{1}{k}}}^k =  \lim_{k \to \infty} |a_k| = L^k
  \end{equation*}
  Let $L < 1$. Note that if $L<1$ there exist some $r$ such that $L<r<1$ and furthermore we know that, $L^k<r^k<1$.
\end{proof}









\end{exercise}
\vspace{.5in}










%Show compactness on the interval [x, x+L], or show that f is bounded and the use the definition of uniform convergence
\begin{exercise}{5} We say that a function $f: \Reals \to \Reals$ is periodic if there is a number $L$ such that $f(x) = f(x + L)$
  for all $x \in \Reals$. Show that a continuous, periodic function is uniformly continuous.\\
  
\end{exercise}
\vspace{.5in}

%http://math341.cardon.byu.edu/solutions341/hw_2.6.pdf
\begin{exercise}{6} Use the Nested interval Property to deduce the Axiom of Completeness without using any other form of the Axiom of
  Completeness.HINT: Look at the proof of the Bolzano-Weierstrass Theorem.\\
  
\end{exercise}
\vspace{.5in}


% Use the density of rationals to get a sequence or rationals to $c$. Note that every sequence of irrational numbers converging to 
% c is sandwiched by a sequence of rationals converging to $c$ thus the function is continuous on irrationals. 
\begin{exercise}{7} Let $(r_n)$ be an enumeration of the rational numbers. Define $f: \Reals \to \Reals$ by 
  \begin{equation*}
    f(x) = 
    \begin{cases} 
      \dfrac{1}{n} & x = r_n \\
      0 & x \not\in \Rats 
   \end{cases}
  \end{equation*}
  Determine, with proof, where $f$ is continuous.\\
  
\begin{proof}
  Suppose $(r_n)$ is an enumeration of the rational numbers, and we define a function $f: \Reals \to \Reals$. 
  Let $c \in \Rats$. Now consider the construction of a sequence $x_n \to c$ where $x_n \not\in \Reals$. Note that the function limit
  of the sequence $f(x) \to 0$ while $f(c) = \frac{1}{n}$ for some $n$. Since $x_n \to c$ and $f(x_n) \not\to f(c)$ we get that by Corollary 4.3.3
  $f$ is not continuous at $c \in \Rats$. \\

  Now let $c \not\in \Rats$. Suppose some sequence $x_n \to c$. Note that $x_n$ is either comprised entirely of rational numbers, or irrational numbers or a mic of both.
  In all cases $f(x_n) \to 0$ and since $f(c) = 0$ by the sequential definition of continuity $f(x)$ is continuous at $c \not\in \Rats$.


\end{proof}






\end{exercise}
\vspace{.5in}







%Take the given definition perform a change of variable and note that how the limits change. 
\begin{exercise}{8} Let $g$ be defined on an interval $A$, and let $c \in A.$\\
  \begin{enumerate}
    \item Explain why $g'(c)$ in Definition 5.2.1 could have been given by,
    \begin{equation*}
      g'(c) = \lim_{h \to 0} \dfrac{g(c+h) - g(c)}{h}.
    \end{equation*}


%https://math.stackexchange.com/questions/576198/prove-that-the-symmetric-derivative-of-a-function-exists-whenever-the-derivative 
    \item  Assume $A$ is open. If $g$ is differentiable at $c \in A$, show
    \begin{equation*}
      g'(c) = \lim_{h \to 0} \dfrac{g(c+h) - g(c - h)}{2h}
    \end{equation*}
  \end{enumerate}
  
\end{exercise}
\vspace{.5in}




%EZ clap definition 
\begin{exercise}{9} Consider the function,
  \begin{equation*}
    f(x) = \sum_{k = 1}^{\infty} \dfrac{1}{2^k} sin(kx).
  \end{equation*}
  Show that $f$ is differentiable.
  
\end{exercise}
\vspace{.5in}




% By contradiction suppose that there is a solution to f(x) = x, By the definition of the derivative we know that,
%f'(x) = 1. Since the f">0 we know that f" is strictly increasing, since f'(1)<1 we know that f'(x)<x
\begin{exercise}{10} Suppose that $f: [0,1] \to \Reals$ is twice differentiable, $f(0) > 0$, $f(1) = 1$, 
  and $f'(1) < 1$. Suppose also that $f'' > 0$ on $[0,1]$. Show that there does not exist a solution of the equation
  $f(x) = x$ in $[0,1)$.
  
\end{exercise}
\vspace{.5in}






%https://www.slader.com/textbook/9781493927128-understanding-analysis-2nd-edition/223/exercises/5/#
\begin{exercise}{11} Assume that, for each $n$, $f_n$ is an integrable function on $[a,b].$
  If $(f_n) \to f$ uniformly on $[a,b]$ prove that $f$ is also integrable on this set. 
  
\end{exercise}
\vspace{.5in}

%https://www.slader.com/textbook/9781493927128-understanding-analysis-2nd-edition/237/exercises/8/#
\begin{exercise}{12} Let,
  \begin{equation*}
    L(x) = \int_1^x \dfrac{1}{t} dt,
  \end{equation*}
  where we consider only $x > 0$.
  \begin{enumerate}
    \item What is $L(1)$? Explain why $L$ is differentiable and find $L'(x).$
    \item Show that $L(xy) = L(x)+L(y)$
    \item Show that $L(x/y) = L(x) - L(y)$
    \item Let,
    \begin{equation*}
      \gamma_n = (1 + \frac{1}{2} + \frac{1}{3}+ \dots+ \frac{1}{n}) - L(n).
    \end{equation*}
    Prove that $(\gamma_n)$ converges. The constant $\gamma = \lim \gamma_n$ is called Euler's constant.
  \end{enumerate}
\end{exercise}



\end{document}