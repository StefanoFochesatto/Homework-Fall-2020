%%%%%%%%%%%%%%%%%%%%%%%%%%%%%%%%%%%%%%%%%%%%%%%%%%%%%%%%%%%%%%%%%%%%%%%%%%%%%%%%%%%%%%%
%%%%%%%%%%%%%%%%%%%%%%%%%%%%%%%%%%%%%%%%%%%%%%%%%%%%%%%%%%%%%%%%%%%%%%%%%%%%%%%%%%%%%%%
% 
% This top part of the document is called the 'preamble'.  Modify it with caution!
%
% The real document starts below where it says 'The main document starts here'.

\documentclass[12pt]{article}
\usepackage{graphicx}
\usepackage{float}
\usepackage{amssymb,amsmath,amsthm}
\usepackage[top=1in, bottom=1in, left=1.25in, right=1.25in]{geometry}
\usepackage{fancyhdr}
\usepackage{enumerate}

% Comment the following line to use TeX's default font of Computer Modern.
\usepackage{times,txfonts}

\newtheoremstyle{homework}% name of the style to be used
  {18pt}% measure of space to leave above the theorem. E.g.: 3pt
  {12pt}% measure of space to leave below the theorem. E.g.: 3pt
  {}% name of font to use in the body of the theorem
  {}% measure of space to indent
  {\bfseries}% name of head font
  {:}% punctuation between head and body
  {2ex}% space after theorem head; " " = normal interword space
  {}% Manually specify head
\theoremstyle{homework} 

% Set up an Exercise environment and a Solution label.
\newtheorem*{exercisecore}{Exercise \@currentlabel}
\newenvironment{exercise}[1]
{\def\@currentlabel{#1}\exercisecore}
{\endexercisecore}

\newcommand{\localhead}[1]{\par\smallskip\noindent\textbf{#1}\nobreak\\}%
\newcommand\solution{\localhead{Solution:}}

%%%%%%%%%%%%%%%%%%%%%%%%%%%%%%%%%%%%%%%%%%%%%%%%%%%%%%%%%%%%%%%%%%%%%%%%
%
% Stuff for getting the name/document date/title across the header
\makeatletter
\RequirePackage{fancyhdr}
\pagestyle{fancy}
\fancyfoot[C]{\ifnum \value{page} > 1\relax\thepage\fi}
\fancyhead[L]{\ifx\@doclabel\@empty\else\@doclabel\fi}
\fancyhead[C]{\ifx\@docdate\@empty\else\@docdate\fi}
\fancyhead[R]{\ifx\@docauthor\@empty\else\@docauthor\fi}
\headheight 15pt

\def\doclabel#1{\gdef\@doclabel{#1}}
\doclabel{Use {\tt\textbackslash doclabel\{MY LABEL\}}.}
\def\docdate#1{\gdef\@docdate{#1}}
\docdate{Use {\tt\textbackslash docdate\{MY DATE\}}.}
\def\docauthor#1{\gdef\@docauthor{#1}}
\docauthor{Use {\tt\textbackslash docauthor\{MY NAME\}}.}
\makeatother

% Shortcuts for blackboard bold number sets (reals, integers, etc.)
\newcommand{\Reals}{\ensuremath{\mathbb R}}
\newcommand{\Nats}{\ensuremath{\mathbb N}}
\newcommand{\Ints}{\ensuremath{\mathbb Z}}
\newcommand{\Rats}{\ensuremath{\mathbb Q}}
\newcommand{\Cplx}{\ensuremath{\mathbb C}}
%% Some equivalents that some people may prefer.
\let\RR\Reals
\let\NN\Nats
\let\II\Ints
\let\CC\Cplx

%%%%%%%%%%%%%%%%%%%%%%%%%%%%%%%%%%%%%%%%%%%%%%%%%%%%%%%%%%%%%%%%%%%%%%%%%%%%%%%%%%%%%%%
%%%%%%%%%%%%%%%%%%%%%%%%%%%%%%%%%%%%%%%%%%%%%%%%%%%%%%%%%%%%%%%%%%%%%%%%%%%%%%%%%%%%%%%
% 
% The main document start here.

% The following commands set up the material that appears in the header.
\doclabel{Stat 300: Homework 3}
\docauthor{Stefano Fochesatto}
\docdate{\today}

\begin{document}

\textbf{2.56:} For any events $A$ and $B$ with $P(B) > 0$, show that $P(A|B) + P(A'|B) = 1$\\

\textbf{Solution:} Using the multiplication rule and set theory,
\begin{align*}
  P(A|B) + P(A'|B) &= \dfrac{P(A \cap B)}{P(B)} + \dfrac{P(A' \cap B)}{P(B)},\\
  &= \dfrac{P(A \cap B) + P(A' \cap B)}{P(B)},\\
  &= \dfrac{P(B)}{P(B)},\\
  &= 1. 
\end{align*}
Thus we hve show that for any events $A$ and $B$ where $P(B) > 0$ then $P(A|B) + P(A'|B) = 1$.
\vspace{1in}







\textbf{2.74:} Tho proportions of blood phenotypes in the U.S population are as follows:
 \begin{center}
\begin{tabular}{ c c c c}
  $A$ & $B$ & $AB$ & $O$ \\
  .40 & .11 & .04 & .45 \\
\end{tabular}
\end{center}
Assuming the phenotypes of two randomly selected individuals are independent of one another, what is the probability that both
phenotypes are $O$? what is the probability that the phenotype of two randomly selected individuals match.\\

\textbf{Solution:} Since the two randomly selected individuals are independent of each other we can simply use the multiplication rule to compute the
probability where both are type $O$,
\begin{equation*}
  P(O\cap O) = P(O)P(O) = .45^2 = .2025.
\end{equation*}
Furthermore we can calculate the probability that the phenotype of two randomly selected individuals match by performing the same operation and summing over each disjoint pair,
\begin{align*}
  P((O\cap O) \cup (A\cap A) \cup (B\cap B) \cup (AB\cap AB)) &= .45^2 + .40^2 + .11^2 +.04^2,\\
  &= .3762.
\end{align*}
\vspace{1in}













\textbf{2.78:} A boiler has five identical relief valves. The probability that any particular valve will open on demand is $.96$.
Assuming independent operation of the valves, calculate the probability that at least one valve opens and the probability that at least one valve fails to open.\\ 

\textbf{Solution:} First note that the probability that at least one valve opens is the same as the compliment of the probability that all valves fail. So since each valve is identical and independent,
\begin{equation*}
  P(\text{one valve opens}) = 1 - P(\text{all valves fail}) = 1 - .04^5 \approx 1.
\end{equation*}
Similarly we can calculate the probability that at least one valve fails,
\begin{equation*}
  P(\text{one valve fails}) = 1 - P(\text{all valves open}) = 1 - .96^5 \approx .1846.
\end{equation*}
\vspace{1in}






\textbf{2.82:} Consider independently rolling two fair dice, one res and the other green. Let $A$ be the event that the red die shows 3 dots, 
$B$ be the event that the green die shows 4 dots, and $C$ be the event that the total number of dots showing on the dice is 7. Are these events pairwise independent?\\

\textbf{Solution:} Recall, that to show two events $A$ and $B$ are independent we must demonstrate $P(A\cup B) = P(A)P(B)$. First we must calculate the sole
probabilities of $A, B$ and $C$. Note that the total number of rolls possible from 2 dice is our sample space, thus $S = 6^2$. Observe that the number of rolls where one of the die are fixed is 6, and therefore,
\begin{equation*}
  P(A) = P(B) = \dfrac{6}{6^2} = \dfrac{1}{6}.
\end{equation*}
Note the number of integer solutions to,
\begin{equation*}
  z_1 + z_2 = 7
\end{equation*}
where $1 \le z_{1,2} le 6$ is 6 and therefore,
\begin{equation*}
  P(6) = \dfrac{6}{6^2} = \dfrac{1}{6}.
\end{equation*}
Also note that any pairwise intersection of $A,B,C$ gives the roll $(3,4)$ in some form thus
\begin{equation*}
  P(A\cap B) = P(A\cap C) = P(C\cap B) =  \dfrac{1}{6^2}.
\end{equation*}
Since $P(A) = P(B) = P(C) = \frac{1}{6}$ we know that any pairwise multiplication of the probabilities will always yield $\frac{1}{6^2}$ thus the events are pairwise independent. 
\vspace{1in}


\textbf{2.88:} The probability that an individual randomly selected from a particular population had a certain disease is $.05$.
A diagnostic test correctly detects the presence of the disease $98\%$ of the time and correctly detects the absence of the disease $99\%$
of the time. If the test is applied twice, independently and return positive, what is the posterior probability that the selected individual had the disease?\\

\textbf{Solution:} First let event $A$ be that an individual has a disease, and let $B$ be the event where an individual tests positive. Using our terms recall what we were given,
\begin{align*}
  P(A) &= .05\\
  P(A') &= 1 - .05 = .95\\
  P(B|A) &= .98\\
  P(B|A') &= .99
\end{align*}
We want to find the probability that an individual has the disease given that they tested positive twice in two independent tests. Through Bayes' Rule we know that
\begin{equation*}
  P(A|BB) = \dfrac{P(BB\cap A)}{P(BB\cap A)+P(BB\cap A')}
\end{equation*} 
Because the tests are independent, by the multiplication rule we know that,
\begin{align*}
  P(BB\cap A) &= P(B \cap A)^2,\\
  &= (P(A)P(B|A))^2,\\
  &=.002401.
\end{align*}
Making the same argument for the  $P(BB\cap A')$ term, 
\begin{align*}
  P(BB\cap A) &= P(B \cap A')^2,\\
  &= (P(A')P(B|A'))^2,\\
  &=.00009025.
\end{align*}
Thus we find that the posterior probability is
\begin{equation*}
  P(A|BB) =\dfrac{P(BB\cap A)}{P(BB\cap A)+P(BB\cap A')} = \dfrac{.002401}{.002401 + .00009025} =.9638
\end{equation*}
\vspace{1in}




\textbf{3.2:} Give three example of bernoulli random variables?\\

\textbf{Solution:} Consider the following,
\begin{enumerate}
  \item The event that the next digit in a binary sequence is a 1.
  \item The event that the next digit in a natural number sequence has even parity.
  \item The event that the next digit in an integer number sequence, excluding zero is negative.
\end{enumerate}
\vspace{1in}







\textbf{3.12:} Airlines somiethimes overbook flights. Suppose that for a plane of 50 seats, 55 passengers have a ticket. Definine the random variables $Y$
as the number  of ticketed passengers who actually show up for the flight. 
\textbf{Solution:}
\vspace{1in}































































































































\vspace{1in}




























\end{document}