%%%%%%%%%%%%%%%%%%%%%%%%%%%%%%%%%%%%%%%%%%%%%%%%%%%%%%%%%%%%%%%%%%%%%%%%%%%%%%%%%%%%%%%
%%%%%%%%%%%%%%%%%%%%%%%%%%%%%%%%%%%%%%%%%%%%%%%%%%%%%%%%%%%%%%%%%%%%%%%%%%%%%%%%%%%%%%%
% 
% This top part of the document is called the 'preamble'.  Modify it with caution!
%
% The real document starts below where it says 'The main document starts here'.

\documentclass[12pt]{article}
\usepackage{graphicx}
\usepackage{float}
\usepackage{amssymb,amsmath,amsthm}
\usepackage[top=1in, bottom=1in, left=1.25in, right=1.25in]{geometry}
\usepackage{fancyhdr}
\usepackage{enumerate}

% Comment the following line to use TeX's default font of Computer Modern.
\usepackage{times,txfonts}

\newtheoremstyle{homework}% name of the style to be used
  {18pt}% measure of space to leave above the theorem. E.g.: 3pt
  {12pt}% measure of space to leave below the theorem. E.g.: 3pt
  {}% name of font to use in the body of the theorem
  {}% measure of space to indent
  {\bfseries}% name of head font
  {:}% punctuation between head and body
  {2ex}% space after theorem head; " " = normal interword space
  {}% Manually specify head
\theoremstyle{homework} 

% Set up an Exercise environment and a Solution label.
\newtheorem*{exercisecore}{Exercise \@currentlabel}
\newenvironment{exercise}[1]
{\def\@currentlabel{#1}\exercisecore}
{\endexercisecore}

\newcommand{\localhead}[1]{\par\smallskip\noindent\textbf{#1}\nobreak\\}%
\newcommand\solution{\localhead{Solution:}}

%%%%%%%%%%%%%%%%%%%%%%%%%%%%%%%%%%%%%%%%%%%%%%%%%%%%%%%%%%%%%%%%%%%%%%%%
%
% Stuff for getting the name/document date/title across the header
\makeatletter
\RequirePackage{fancyhdr}
\pagestyle{fancy}
\fancyfoot[C]{\ifnum \value{page} > 1\relax\thepage\fi}
\fancyhead[L]{\ifx\@doclabel\@empty\else\@doclabel\fi}
\fancyhead[C]{\ifx\@docdate\@empty\else\@docdate\fi}
\fancyhead[R]{\ifx\@docauthor\@empty\else\@docauthor\fi}
\headheight 15pt

\def\doclabel#1{\gdef\@doclabel{#1}}
\doclabel{Use {\tt\textbackslash doclabel\{MY LABEL\}}.}
\def\docdate#1{\gdef\@docdate{#1}}
\docdate{Use {\tt\textbackslash docdate\{MY DATE\}}.}
\def\docauthor#1{\gdef\@docauthor{#1}}
\docauthor{Use {\tt\textbackslash docauthor\{MY NAME\}}.}
\makeatother

% Shortcuts for blackboard bold number sets (reals, integers, etc.)
\newcommand{\Reals}{\ensuremath{\mathbb R}}
\newcommand{\Nats}{\ensuremath{\mathbb N}}
\newcommand{\Ints}{\ensuremath{\mathbb Z}}
\newcommand{\Rats}{\ensuremath{\mathbb Q}}
\newcommand{\Cplx}{\ensuremath{\mathbb C}}
%% Some equivalents that some people may prefer.
\let\RR\Reals
\let\NN\Nats
\let\II\Ints
\let\CC\Cplx

%%%%%%%%%%%%%%%%%%%%%%%%%%%%%%%%%%%%%%%%%%%%%%%%%%%%%%%%%%%%%%%%%%%%%%%%%%%%%%%%%%%%%%%
%%%%%%%%%%%%%%%%%%%%%%%%%%%%%%%%%%%%%%%%%%%%%%%%%%%%%%%%%%%%%%%%%%%%%%%%%%%%%%%%%%%%%%%
% 
% The main document start here.

% The following commands set up the material that appears in the header.
\doclabel{Stat 300: Homework 6}
\docauthor{Stefano Fochesatto}
\docdate{\today}

\begin{document}
\begin{enumerate}



\item\hspace{.5in}\textbf{Exercise 4.28:} Let Z be a standard normal random variable and calculate the following probabilities, drawing pictures wherever appropriate.
\begin{enumerate}
\item $P(0\le Z\le2.17)$
\item $P(0\le Z\le1)$
\addtocounter{enumii}{2}
\item $P(Z\le 1.37)$
\addtocounter{enumii}{4}
\item $P(|Z| \le 2.50)$
\end{enumerate}

\textbf{Answer:} Note that for all of the standard normal probabilities we have an appendix that gives us the value of the
cdf $P(Z\le x) = \phi(x)$ for up to values of $x$ in the hundredths. Note that,
\begin{equation*}
  P(a \le Z \le b) = \phi(b) - \phi(a).
\end{equation*}

\begin{enumerate}
\item Calculating using Table A-3,
\begin{equation*}
  P(0\le Z\le2.17) = \phi(2.17) - \phi(0) = .9850 - .50 = .4850.
\end{equation*}


\item Similarly,
\begin{equation*}
  P(0\le Z\le1) = \phi(1) - \phi(0) = .8413 - .50 = .3413.
\end{equation*}
\addtocounter{enumii}{2}
\item Similarly,
\begin{equation*}
  P(Z\le 1.37) = \phi(1.37) = .9147.
\end{equation*}
\addtocounter{enumii}{4}
\item Simplifying the inequality to remove the absolute value,
\begin{equation*}
  P(|Z| \le 2.50) = P(-2.50 \le Z \le 2.50).
\end{equation*}
Calculating with table A-3,
\begin{equation*}
  P(-2.50 \le Z \le 2.50) = \phi(2.50) - \phi(-2.50) = .9938 - .0062 = .9876
\end{equation*}
\end{enumerate}

\vspace{.5in}





\item\hspace{.5in}\textbf{Exercise 4.34:} The article “Reliability of Domestic­Waste Biofilm Reactors” suggests that substrate concentration $(mg/cm^3)$ of influent to a reactor is normally distributed with $\mu = .30$ and $\sigma = .06$.
\begin{enumerate}
\item What is the probability that the concentration exceeds $.50$?
\item What is the probability that the concentration is at most $.20$?
\item How would you characterize the largest $5\%$ of all concentration values?
\end{enumerate}
\textbf{Answer:} 
\begin{enumerate}
\item Note that $X$ is a non-standard normal distribution, however we can calculate the probability as a standard normal
after computing the standardized variable. Computing the standardized variable with, $X = .50$, $\mu = .30$ and $\sigma = .06$,
\begin{equation*}
  P(X>.5) = P(Z>\frac{.5 - .30}{.06}) = P(Z > 3.33) = 1-\phi(3.33) = .0004.
\end{equation*}
\item Similarly computing $P(X<.20)$,
\begin{equation*}
  P(X\le.20) = P(Z \le \frac{.2 - .30}{.06}) = P(Z \le -1.67) = \phi(-1.67) = .475 
\end{equation*}
\item Recall that we denote the largest $5\%$ of all concentrations values can be denoted as all values greater than the critical value of $Z_{.05}$. 
Note that our cdf will calculates values up to a certain point, therefore
\begin{equation*}
  \phi(Z_{0.05}) = 1 - 0.05 = .95.
\end{equation*}
Since we have $Z_{0.05} = 1.645$ from table A-3 all we have to do is set it equal to the standardizing variable to solve for $X_{0.05}$,
\begin{align*}
  \frac{X_{0.05} - .30}{.06} &= 1.645\\
  X_{0.05} &= (1.645)(.06)+.3\\
  X_{0.05} &= .3987.
\end{align*}
Therefore the top $5\%$ of all concentrations occur when $X\geq .3987$.
\end{enumerate}
\vspace{.5in}







\item\hspace{.5in}\textbf{Exercise 4.50:} In response to concerns about nutritional contents of fast foods, McDonald’s has announced that it will use a new cooking oil for its french fries that will decrease substantially trans fatty acid levels and increase the amount of more beneficial polyunsaturated fat. The company claims that $97$ out of $100$ people cannot detect a difference in taste between the new and old oils. Assuming that this figure is correct (as a long-run proportion), what is the approximate probability that in a random sample of $1000$ individuals who have purchased fries at McDonald’s,
\begin{enumerate}
\item At least $40$ can taste the difference between the two oils?
\item At most $5\%$ can taste the difference between the two oils?
\end{enumerate}
\textbf{Answer:} 
\begin{enumerate}
\item We will Continue by using a standard normal distribution to approximate the Binomial r.v., $X \sim Binom(1000, .97)$ where $X$ is the number of people who can't taste the difference between oils. 
Checking if $X$ satisfies approximation criteria,
\begin{equation*}
  \mu = 1000(.97) = 970 \geq 10, 
\end{equation*}
\begin{equation*}
  nq = 1000(.03) = 30 \geq 10 .
\end{equation*}
Calculating $\sigma$ for our approximate normal,
\begin{equation*}
  \sigma = \sqrt{1000(.97)(.03)} = 5.39.
\end{equation*}
Since at most 40 people tasting the difference is the same as at least 960 people not tasting the difference we know that $P(X\geq 40) = P(X \le 960)$,
\begin{equation*}
  P(X \le 960) \approx P(Z \le \frac{960 + .5 - 970}{5.39}) = \phi(-1.76) = .039.
\end{equation*}
\item Similarly we approximate. Note that $5\%$ of 1000 is 50 and saying at most 50 people can detect the difference is the sme as saying at least 950 cannot detect the difference. Hence,
\begin{equation*}
  P(X \geq 950) = 1 - P(X < 950) \approx 1 - P(Z < \dfrac{950 + .5 - 970}{5.39}) = 1 - \phi(-3.617) = .9998.
\end{equation*}
\end{enumerate}
\vspace{.5in}




\item\hspace{.5in}\textbf{Exercise 4.38:} There are two machines available for cutting corks intended for use in wine bottles. The first produces corks with diameters that are normally distributed with mean 3 cm and standard deviation $.1$ cm. The second machine produces corks with diameters that have a normal distribution with mean $3.04$ cm and standard deviation $.02$ cm. Acceptable corks have diameters between $2.9$ cm and $3.1$ cm. Which machine is more likely to produce an acceptable cork?\\
\\
\textbf{Answer:} Consider the r.v. $X_1 \sim norm(3, .1)$ and  $X_2 \sim norm(3.04, .02)$ where $X_i$ is the diameter of the cork produced by the $ith$ machine. We will proceed by 
using the standard normal to calculate $P(2.9 \le X_i \le 3.1)$. Consider the first machine,
\begin{equation*}
  P(2.9 \le X_1 \le 3.1) = P(\frac{2.9 - 3}{.1} \le Z \le \frac{3.1 - 3}{.1}) = P(-1 \le Z \le 1) = \phi(1) - \phi(-1) = .6826.
\end{equation*} 
Similarly we calculate the probability for the second machine,
\begin{equation*}
  P(2.9 \le X_2 \le 3.1) = P(\frac{2.9 - 3.04}{.02} \le Z \le \frac{3.1 - 3.04}{.02}) = P(-7 \le Z \le 3) = \phi(3) - \phi(-7) = .9987. 
\end{equation*}
Since machine 2 has a higher probability it is more likely to produce cork within the given range.



\vspace{.5in}




\item\hspace{.5in}\textbf{Exercise 4.42:} The temperature reading from a thermocouple placed in a constant-temperature medium is normally distributed with mean $\mu$, the actual temperature of the medium, and standard deviation $\sigma$. What would the value of $\sigma$ have to be to ensure that $95\%$ of all readings are within $.18$ of $\mu$?\\
\\
\textbf{Answer:} Recall that by the empirical rule $95\%$ of a normal distribution must lie within 2 standard deviations of the mean. Hence,
\begin{align*}
  \mu + 2\sigma &= \mu + .18\\
  \sigma &= .09.
\end{align*} 
\vspace{.5in}




\item\hspace{.5in}\textbf{Exercise 4.60:} Let $X$ denote the distance (m) that an animal moves from its birth site to the first territorial vacancy it encounters. Suppose that for banner-tailed kangaroo rats, $X$ has an exponential distribution with parameter $\lambda = .01386$,
\begin{enumerate}
\item What is the probability that the distance is at most $100$ m? At most $200$ m? Between $100$ and $200$ m?
\item What is the probability that distance exceeds the mean distance by more than $2$ standard deviations?
\item What is the value of the median distance?
\end{enumerate}
\textbf{Answer:} First note that the pdf and cdf for an exponential distribution with $\lambda =  .01386$ are,

\[ f(x) =  \begin{cases} 
    (\lambda)e^{-\lambda(x)} & x \geq 0 \\
    0 & otherwise 
 \end{cases}
\]

\[ F(x) =\begin{cases} 
    1-e^{-\lambda(x)} & x \geq 0 \\
    0 & otherwise 
 \end{cases}
\]

\begin{enumerate}
\item Calculating the three probabilities using the given cdf,
\begin{equation*}
  P(X \le 100) = F(100) =  1-e^{-\lambda(100)} = .7499.
\end{equation*}
\begin{equation*}
  P(X \le 200) = F(200) =  1-e^{-\lambda(200)} = .9377.
\end{equation*}
\begin{equation*}
  P(100 \le X \le 200) = F(200) - F(100) = .9377 - .7499 = .1878.
\end{equation*}

\item Recall that for an exponential distribution the mean and standard deviation are the same and are calculated,
\begin{equation*}
  \mu = \sigma  = \frac{1}{\lambda}.
\end{equation*}
Therefore calculating the probability which the distance exceeds the mean by more 2 standard deviations,
\begin{equation*}
  P(X > \mu + 2\sigma) = P(X > 216.54) =1 - P(X \le 216.54)   = 1 - F(216.45) = 1 - (1-e^{-\lambda(216.45)}) = .0498.
\end{equation*}
\item Recall that the median is located at the point where the cdf is equal to .5. Thus,
\begin{align*}
  F(\widetilde{x}) &= .5\\
  1-e^{-\lambda(\widetilde{x})}&= .5\\
  e^{-\lambda(\widetilde{x})}&= .5\\
  \widetilde{x}&= \frac{ln(.5)}{-\lambda}\\
  \widetilde{x}&= 50.
\end{align*}
\end{enumerate}
\vspace{.5in}




\end{enumerate}
\end{document}