%%%%%%%%%%%%%%%%%%%%%%%%%%%%%%%%%%%%%%%%%%%%%%%%%%%%%%%%%%%%%%%%%%%%%%%%%%%%%%%%%%%%%%%
%%%%%%%%%%%%%%%%%%%%%%%%%%%%%%%%%%%%%%%%%%%%%%%%%%%%%%%%%%%%%%%%%%%%%%%%%%%%%%%%%%%%%%%
% 
% This top part of the document is called the 'preamble'.  Modify it with caution!
%
% The real document starts below where it says 'The main document starts here'.

\documentclass[12pt]{article}
\usepackage{graphicx}
\usepackage{float}
\usepackage{amssymb,amsmath,amsthm}
\usepackage[top=1in, bottom=1in, left=1.25in, right=1.25in]{geometry}
\usepackage{fancyhdr}
\usepackage{enumerate}

% Comment the following line to use TeX's default font of Computer Modern.
\usepackage{times,txfonts}

\newtheoremstyle{homework}% name of the style to be used
  {18pt}% measure of space to leave above the theorem. E.g.: 3pt
  {12pt}% measure of space to leave below the theorem. E.g.: 3pt
  {}% name of font to use in the body of the theorem
  {}% measure of space to indent
  {\bfseries}% name of head font
  {:}% punctuation between head and body
  {2ex}% space after theorem head; " " = normal interword space
  {}% Manually specify head
\theoremstyle{homework} 

% Set up an Exercise environment and a Solution label.
\newtheorem*{exercisecore}{Exercise \@currentlabel}
\newenvironment{exercise}[1]
{\def\@currentlabel{#1}\exercisecore}
{\endexercisecore}

\newcommand{\localhead}[1]{\par\smallskip\noindent\textbf{#1}\nobreak\\}%
\newcommand\solution{\localhead{Solution:}}

%%%%%%%%%%%%%%%%%%%%%%%%%%%%%%%%%%%%%%%%%%%%%%%%%%%%%%%%%%%%%%%%%%%%%%%%
%
% Stuff for getting the name/document date/title across the header
\makeatletter
\RequirePackage{fancyhdr}
\pagestyle{fancy}
\fancyfoot[C]{\ifnum \value{page} > 1\relax\thepage\fi}
\fancyhead[L]{\ifx\@doclabel\@empty\else\@doclabel\fi}
\fancyhead[C]{\ifx\@docdate\@empty\else\@docdate\fi}
\fancyhead[R]{\ifx\@docauthor\@empty\else\@docauthor\fi}
\headheight 15pt

\def\doclabel#1{\gdef\@doclabel{#1}}
\doclabel{Use {\tt\textbackslash doclabel\{MY LABEL\}}.}
\def\docdate#1{\gdef\@docdate{#1}}
\docdate{Use {\tt\textbackslash docdate\{MY DATE\}}.}
\def\docauthor#1{\gdef\@docauthor{#1}}
\docauthor{Use {\tt\textbackslash docauthor\{MY NAME\}}.}
\makeatother

% Shortcuts for blackboard bold number sets (reals, integers, etc.)
\newcommand{\Reals}{\ensuremath{\mathbb R}}
\newcommand{\Nats}{\ensuremath{\mathbb N}}
\newcommand{\Ints}{\ensuremath{\mathbb Z}}
\newcommand{\Rats}{\ensuremath{\mathbb Q}}
\newcommand{\Cplx}{\ensuremath{\mathbb C}}
%% Some equivalents that some people may prefer.
\let\RR\Reals
\let\NN\Nats
\let\II\Ints
\let\CC\Cplx

%%%%%%%%%%%%%%%%%%%%%%%%%%%%%%%%%%%%%%%%%%%%%%%%%%%%%%%%%%%%%%%%%%%%%%%%%%%%%%%%%%%%%%%
%%%%%%%%%%%%%%%%%%%%%%%%%%%%%%%%%%%%%%%%%%%%%%%%%%%%%%%%%%%%%%%%%%%%%%%%%%%%%%%%%%%%%%%
% 
% The main document start here.

% The following commands set up the material that appears in the header.
\doclabel{Stat 300: Homework 2}
\docauthor{Brons Gerrish}
\docdate{Due: September 11th, 2020}

\begin{document}
\begin{enumerate}



\item\hspace{.5in}\textbf{Exercise 3.30:} An individual who has automobile insurance from a certain company is randomly selected. Let $Y$ be the number of moving violations for which the individual was cited during the last $3$ years. The $pmf$ of $Y$ is,

\begin{enumerate}
\item Compute $E(Y)$.
\item Suppose an individual with $Y$ violations incurs a
surcharge of $\$100Y^2$ . Calculate the expected amount
of the surcharge.
\end{enumerate}

\textbf{Answer:} 
\begin{enumerate}
\item $E(Y)=.6(0)+.25(1)+.1(2)+.05(3)=.6$
\item $100(.6)^2=\$36$
\end{enumerate}

\vspace{.5in}





\item\hspace{.5in}\textbf{Exercise 3.34:} Suppose that the number of plants of a particular type found in a rectangular sampling region (called a quadrat by ecologists) in a certain geographic area is an $rv$ $X$ with $pmf$,
\item 
Is E(X) finite? Justify your answer (this is another distribution that statisticians would call heavy-tailed).\\

\textbf{Answer:} 
\vspace{.5in}







\item\hspace{.5in}\textbf{Exercise 3.48:} NBC News reported on May 2, 2013, that $1$ in $20$ children in the United States have a food allergy of some sort. Consider selecting a random sample of $25$ children and let $X$ be the number in the sample who have a food allergy. Then $X\sim Bin(25, .05)$.
\begin{enumerate}
\item Determine both $P(X \le 3)$ and $P(X < 3)$.
\item Determine $P(X \ge 4)$.
\item Determine $P(1 \le X \le 3)$.
\item What are $E(X)$ and $\sigma_X$?
\item In a sample of $50$ children, what is the probability that none has a food allergy?
\end{enumerate}

\textbf{Answer:} 
\begin{enumerate}
\item $P(X\le3)=P(X=3)+P(X=2)+P(X=1)+P(X=0)={25\choose 3}(.05)^3(.95)^{22}+{25\choose 2}(.05)^2(.95)^{23}+{25\choose 1}(.05)(.95)^{24}+{25\choose 0}(.95){25}=0.093+0.231+0.365+0.277=.966$\\
\\
$P(X\le 3)=P(X=2)+P(X=1)+P(X=0)=0.231+0.365+0.277=.873$.
\item $P(X\ge 4)=1-P(X<3)=1-.873=.127$
\item $P(1\le X\le3)=P(X=1)+P(X=2)+P(X=3)=0.093+0.231+0.365=.689$.
\item $E(X)=np=25(.05)=1.25$, $\sigma_X=\sqrt{np(1-p)}=\sqrt{25(.05)(.95)}=\sqrt{1.25(.95)}=\sqrt{1.1875}\approx1.0897$
\item $.95^{50}=.077$.
\end{enumerate}
\vspace{.5in}







\item\hspace{.5in}\textbf{Exercise 3.56:} The College Board reports that $2\%$ of the $2$ million high school students who take the SAT each year receive special accommodations because of documented disabilities \emph{(Los Angeles Times, July 16, 2002)}. Consider a random sample of $25$ students who have recently taken the test.
\begin{enumerate}
\item What is the probability that exactly 1 received a special accommodation?
\item What is the probability that at least 1 received a special accommodation?
\addtocounter{enumii}{1}
\item What is the probability that the number among the $25$ who received a special accommodation is within $2$ standard deviations of the number you would expect to be accommodated?
\end{enumerate}

\textbf{Answer:} 
\begin{enumerate}
\item ${25 \choose 1}(.02)(.98)^{24}=.308$.
\item $1-P(X<1)=1-{25 \choose 0}(.02)^0(.98)^{25}=1-.603=.397$.
\addtocounter{enumii}{1}
\item We would expect $(.02)(25)=.5$ to be accommodated. Two standard deviations is $2\sqrt{(25)(.02)(.98)}=1.4$. Thus we want $P(0\le X\le 1.9)=P(0\le X\le 1)=P(X=0)+P(X=1)=.603+.308=.911$
\end{enumerate}
\vspace{.5in}







\item\hspace{.5in}\textbf{Exercise 3.58:} A very large batch of components has arrived at a distributor. The batch can be characterized as acceptable only if the proportion of defective components is at most $.10$. The distributor decides to randomly select $10$ components and to accept the batch only if the number of defective components in the sample is at most $2$.
\begin{enumerate}
\item What is the probability that the batch will be accepted when the actual proportion of defectives is $.01$? $.05$? $.10$? $.20$? $.25$?
\item Let p denote the actual proportion of defectives in the batch. A graph of $P$(batch is accepted) as a function of $p$, with $p$ on the horizontal axis and $P$(batch is accepted) on the vertical axis, is called the operating characteristic curve for the acceptance sampling plan. Use the results of part $(a)$ to sketch this curve for $0 \le p \le 1$.
\item Repeat parts $(a)$ and $(b)$ with “$1$” replacing “$2$” in the acceptance sampling plan.
\end{enumerate}
\textbf{Answer:} 
\begin{enumerate}
\item For $.01$, $P(X\le 2)=P(X=0)+P(X=1)+P(X=2)={10\choose 0}(.01)^0(.99)^10+{10\choose 1}(.01)^1(.99)^9+{10\choose 2}(.01)^2(.99)^8=.904+.091+.004=.999$.\\
\\
For $.05$, $P(X\le 2)=.988$\\
For $.10$, $P(X\le 2)=.930$\\
For $.20$, $P(X\le 2)=.678$\\
For $.25$, $P(X\le 2)=.526$\\
\item 


\item
For $.01$, $P(X\le 1)=.996$.\\
For $.05$, $P(X\le 1)=.914$.\\
For $.10$, $P(X\le 1)=.736$.\\
For $.20$, $P(X\le 1)=.376$.\\
For $.25$, $P(X\le 1)=.244$.\\
For $.50$, $P(X\le 1)=.011$.\\
For $.75$, $P(X\le 1)=.00003$.\\

\end{enumerate}
\vspace{.5in}








\item\hspace{.5in}\textbf{Exercise 3.60:} A toll bridge charges $\$1.00$ for passenger cars and $\$2.50$ for other vehicles. Suppose that during daytime hours, $60\%$ of all vehicles are passenger cars. If $25$ vehicles cross the bridge during a particular daytime period, what is the resulting expected toll revenue? [Hint: Let $X= $ the number of passenger cars; then the toll revenue $h(X)$ is a linear function of $X$.]\\
\\
\textbf{Answer:} \\
$25(.6)(1)+25(.4)(2.5)=\$40$. The first product is the revenue off passenger cars and the second product is for other vehicles. These revenues are found by multiplying the expected value by the charge.
\vspace{.5in}

\item\hspace{.5in}\textbf{Supplemental 1:} Suppose that we get $100$ pairs of patients and want to compare two treatments (this is called a matched pairs design).\\

We will apply one treatment to one patient in a pair and the other treatment to the other patient in a pair (random selection within pairs).\\

If both treatments work equally well, the number of pairs where treatment one is better than treatment two should be $X \sim Binomial(100,0.5)$, because in this case it’s essentially a coin flip. What is the expected value and standard deviation of $X$? What range of values is within $2$ standard deviations of the mean?\\
\\
\textbf{Answer:} 
$E(X)=np=100(.5)=50$\\
\vspace{.5in}

\item\hspace{.5in}\textbf{Exercise 3.62:}
\begin{enumerate}
\item For fixed $n$, are there values of $p(0 \le p \le 1)$ for which $V(X) = 0$? Explain why this is so.
\item For what value of $p$ is $V(X)$ maximized? [Hint: Either graph $V(X)$ as a function of $p$ or else take a derivative.]
\end{enumerate}
\textbf{Answer:} 
\begin{enumerate}
\item
\item
\end{enumerate}
\vspace{.5in}

\end{enumerate}
\end{document}