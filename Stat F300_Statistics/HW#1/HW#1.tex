%%%%%%%%%%%%%%%%%%%%%%%%%%%%%%%%%%%%%%%%%%%%%%%%%%%%%%%%%%%%%%%%%%%%%%%%%%%%%%%%%%%%%%%
%%%%%%%%%%%%%%%%%%%%%%%%%%%%%%%%%%%%%%%%%%%%%%%%%%%%%%%%%%%%%%%%%%%%%%%%%%%%%%%%%%%%%%%
% 
% This top part of the document is called the 'preamble'.  Modify it with caution!
%
% The real document starts below where it says 'The main document starts here'.

\documentclass[12pt]{article}

\usepackage{amssymb,amsmath,amsthm}
\usepackage[top=1in, bottom=1in, left=1.25in, right=1.25in]{geometry}
\usepackage{fancyhdr}
\usepackage{enumerate}

% Comment the following line to use TeX's default font of Computer Modern.
\usepackage{times,txfonts}

\newtheoremstyle{homework}% name of the style to be used
  {18pt}% measure of space to leave above the theorem. E.g.: 3pt
  {12pt}% measure of space to leave below the theorem. E.g.: 3pt
  {}% name of font to use in the body of the theorem
  {}% measure of space to indent
  {\bfseries}% name of head font
  {:}% punctuation between head and body
  {2ex}% space after theorem head; " " = normal interword space
  {}% Manually specify head
\theoremstyle{homework} 

% Set up an Exercise environment and a Solution label.
\newtheorem*{exercisecore}{Exercise \@currentlabel}
\newenvironment{exercise}[1]
{\def\@currentlabel{#1}\exercisecore}
{\endexercisecore}

\newcommand{\localhead}[1]{\par\smallskip\noindent\textbf{#1}\nobreak\\}%
\newcommand\solution{\localhead{Solution:}}

%%%%%%%%%%%%%%%%%%%%%%%%%%%%%%%%%%%%%%%%%%%%%%%%%%%%%%%%%%%%%%%%%%%%%%%%
%
% Stuff for getting the name/document date/title across the header
\makeatletter
\RequirePackage{fancyhdr}
\pagestyle{fancy}
\fancyfoot[C]{\ifnum \value{page} > 1\relax\thepage\fi}
\fancyhead[L]{\ifx\@doclabel\@empty\else\@doclabel\fi}
\fancyhead[C]{\ifx\@docdate\@empty\else\@docdate\fi}
\fancyhead[R]{\ifx\@docauthor\@empty\else\@docauthor\fi}
\headheight 15pt

\usepackage{listings}
\usepackage{color}

\definecolor{dkgreen}{rgb}{0,0.6,0}
\definecolor{gray}{rgb}{0.5,0.5,0.5}
\definecolor{mauve}{rgb}{0.58,0,0.82}


\def\doclabel#1{\gdef\@doclabel{#1}}
\doclabel{Use {\tt\textbackslash doclabel\{MY LABEL\}}.}
\def\docdate#1{\gdef\@docdate{#1}}
\docdate{Use {\tt\textbackslash docdate\{MY DATE\}}.}
\def\docauthor#1{\gdef\@docauthor{#1}}
\docauthor{Use {\tt\textbackslash docauthor\{MY NAME\}}.}
\makeatother

% Shortcuts for blackboard bold number sets (reals, integers, etc.)
\newcommand{\Reals}{\ensuremath{\mathbb R}}
\newcommand{\Nats}{\ensuremath{\mathbb N}}
\newcommand{\Ints}{\ensuremath{\mathbb Z}}
\newcommand{\Rats}{\ensuremath{\mathbb Q}}
\newcommand{\Cplx}{\ensuremath{\mathbb C}}
%% Some equivalents that some people may prefer.
\let\RR\Reals
\let\NN\Nats
\let\II\Ints
\let\CC\Cplx

%%%%%%%%%%%%%%%%%%%%%%%%%%%%%%%%%%%%%%%%%%%%%%%%%%%%%%%%%%%%%%%%%%%%%%%%%%%%%%%%%%%%%%%
%%%%%%%%%%%%%%%%%%%%%%%%%%%%%%%%%%%%%%%%%%%%%%%%%%%%%%%%%%%%%%%%%%%%%%%%%%%%%%%%%%%%%%%
% 
% The main document start here.

% The following commands set up the material that appears in the header.
\doclabel{Stat 300: Homework 1}
\docauthor{Stefano Fochesatto}
\docdate{\today}

\begin{document}





\begin{exercise}{1.14} The accompanying data set consists of observations on shower-flow rate for a sample of $n = 129$ houses in Perth, Australia
  \begin{enumerate}
    \item[\textbf{a.}] Construct a stem-and-leaf display of the data.\\
    \textbf{Solution:}
    \begin{lstlisting}
      > stem(x)

      The decimal point is at the |
    
       2 | 23
       3 | 2344567789
       4 | 01356889
       5 | 00001114455666789
       6 | 0000122223344456667789999
       7 | 00012233455555668
       8 | 02233448
       9 | 012233335666788
      10 | 2344455688
      11 | 2335999
      12 | 37
      13 | 8
      14 | 36
      15 | 0035
      16 | 
      17 | 
      18 | 9    
    \end{lstlisting}

    \vspace{.5in}
     
    \item[\textbf{b.}] What is a typical, or representative flow rate?\\
    \textbf{Solution:}
    
    \begin{lstlisting}
    > fivenum(x)
 Minimum   Lower Quartile  Median  Upper Quartile  Maximum 
     2.2        5.6        7.0        9.6       18.9 
      
    > mean(x)
      [1] 7.707752
  
    > sd(x)
      [1] 3.076844
    \end{lstlisting}
    I would say that a representative flow rate would be closer to the median at $7.0$ since there is an outlier in $18.9$ that is more than 3 standard deviations away from the mean.
    \vspace{.5in}
     
    \item[\textbf{c.}] Does the display appear to be highly concentrated or spread out?\\
    
    \textbf{Solution:} Generally speaking this display appears to be highly concentrated with a small slight skew towards the right.
    \vspace{.5in}
     

    \item[\textbf{d}] Does the distribution of values appear to be reasonably symmetric?If not, how would you describe the departure from symmetry?\\
    
    \textbf{Solution:} The data is does not appear to be symmetrical, since it seems to have a slight positive skew.
    \vspace{.5in}


    \item[\textbf{e}] Would you describe any observation as being far from the rest of the data (an outlier)?\\
    
    \textbf{Solution:} I would describe the data point $x = 18.9$ as an outlier, since it is almost 4 standard deviations from the mean.  
    \vspace{.5in}
     

  \end{enumerate}
\end{exercise}
\vspace{1in}



\begin{exercise}{1.18} Every corporation has a governing board og directors. The number of individuals on a board varies from one corporation to another. One of the authors of the article provided the accompanying data on the number of directors on each board in a random sample of 204 corporations.\\
  \begin{enumerate}
    \item[\textbf{a.}] Construct a histogram of the data based on relative frequencies and comment on any interesting features?\\
    \textbf{Solution:}

    \vspace{.5in}
     
    \item[\textbf{c.}]
    \textbf{Solution:}
    \vspace{.5in}
     
  \end{enumerate}


\end{exercise}
\vspace{1in}


\begin{exercise}{1.22}
\end{exercise}
\vspace{1in}


\begin{exercise}{1.38}
\end{exercise}
\vspace{1in}


\begin{exercise}{1.42}
\end{exercise}
\vspace{1in}


\begin{exercise}{1.44}
\end{exercise}
\vspace{1in}


\begin{exercise}{1.50}
\end{exercise}
\vspace{1in}


\begin{exercise}{1.56}
\end{exercise}
\vspace{1in}



\begin{exercise}{2.4}
\end{exercise}
\vspace{1in}


\begin{exercise}{2.9}
\end{exercise}
\vspace{1in}

\end{document}
